\stylesection{\styleamericanpaleale}

% -----------------------------------------------------------------------------
\begin{recipie}{3 Floyds Alpha King Clone}
% -----------------------------------------------------------------------------

\begin{aboutblock}
Alpha King, a bold, citrus-forward American pale ale has been consistently ranked as
one of the best pale ales made in America. If you're a fan of the versatile Centennial
hop, this beer from Munster, Ind.'s 3 Floyds Brewing Co. is for you!
\end{aboutblock}

\specifications{\styleamericanpaleale}{\galtol{5}}{1.059}{1.014}{6.4~\%}{69}{}{60~min}{}

\begin{methodandtiming}
 
\begin{mashsteps}
\mashstep{\ftoc{154}}{}
\end{mashsteps}

\begin{fermentationsteps}
\fermentationstep{\ftoc{68}}{7~days}
\end{fermentationsteps}

\end{methodandtiming}

\pagebreak

\begin{ingredientsblock}

\begin{malts}
\malt{Pale}{\lbtokg{10}}
\malt{Simpsons Crystal Medium}{\lbtokg{1}}
\malt{Dingemans Cara 45 L}{\lbtokg{0.5}}
\end{malts}

\begin{hops}
\hop{\hopcolumbus}{15.5~\%}{60~min}{\oztog{1}}
\hop{\hopwarrior}{17~\%}{30~min}{\oztog{0.5}}
\hop{\hopcentennial}{10.5~\%}{10~min}{\oztog{1}}
\hop{\hopwarrior}{17~\%}{\dryh{}{7~days}}{\oztog{0.5}}
\hop{\hopcentennial}{10.5~\%}{\dryh{}{7~days}}{\oztog{0.5}}
\end{hops}

\begin{yeasts}
\yeast{Wyeast 1056}
\end{yeasts}

\end{ingredientsblock}

\end{recipie}

% -----------------------------------------------------------------------------
\begin{recipie}{Biloxi Brewing Company Pale Ale}
% -----------------------------------------------------------------------------

\begin{aboutblock}
This is Biloxi Brewing Company's sessionable pale ale that bursts with grapefruit
flavors and aromas from the generous amount of Citra hops used, especially in the
dry hop addition.
\end{aboutblock}

\specifications{\styleamericanpaleale}{\galtol{5}}{1.049}{1.009}{5.2~\%}{37}{}{60~min}{}

\begin{methodandtiming}
 
\begin{mashsteps}
\mashstep{\ftoc{150}}{75~min}
\mashstep{\ftoc{168}}{10~min}
\end{mashsteps}

\begin{fermentationsteps}
\fermentationstep{\ftoc{67}}{}
\end{fermentationsteps}

\begin{directions}
Age at \ftoc{65} for 30 days.
\end{directions}

\end{methodandtiming}

\pagebreak

\begin{ingredientsblock}

\begin{malts}
\malt{Pale}{\lbtokg{6.47}}
\malt{Munich}{\lbtokg{1}}
\malt{Carapils}{\oztokg{12.4}}
\malt{Caramel / Crystal 40 L}{\oztokg{12.4}}
\end{malts}

\begin{hops}
\hop{\hopcitra}{14~\%}{60~min}{\oztog{0.42}}
\hop{\hopcascade}{6.3~\%}{15~min}{\oztog{0.42}}
\hop{\hopcitra}{12.2~\%}{15~min}{\oztog{0.42}}
\hop{\hopcascade}{}{\whirl{}{}}{\oztog{0.42}}
\hop{\hopcitra}{}{\whirl{}{}}{\oztog{0.42}}
\hop{\hopcitra}{}{\dryh{}{}}{\oztog{0.83}}
\hop{\hopcascade}{}{\dryh{}{}}{\oztog{0.42}}

\end{hops}

\begin{yeasts}
\yeast{Fermentis SafAle US-05}
\end{yeasts}

\end{ingredientsblock}

\end{recipie}

% -----------------------------------------------------------------------------
\begin{recipie}{Citra Summit American Pale Ale}
% -----------------------------------------------------------------------------

\begin{aboutblock}
Ben Gaylord of Denver, CO, won a bronze medal in Category \#10: American Pale Ale
during the 2019 National Homebrew Competition Final Round in Providence, RI.
Gaylord's American Pale Ale was chosen as a top three entry among 312 entries in
the category.
\end{aboutblock}

\specifications{\styleamericanpaleale}{\galtol{5.5}}{1.067}{1.012}{5.7~\%}{77.4}{\srmtoebc{7}}{60~min}{2.6}

\begin{methodandtiming}
 
\begin{mashsteps}
\mashstep{\ftoc{154}}{60~min}
\end{mashsteps}

\begin{fermentationsteps}
\fermentationstep{\ftoc{68}}{6~days}
\fermentationstep{\ftoc{72}}{Free-raise to}
\fermentationstep{\ftoc{72}}{7~days}
\end{fermentationsteps}

\begin{directions}
On day 7, add dry hops and allow temperature to free rise to \ftoc{72}.
\end{directions}

\end{methodandtiming}

\pagebreak

\begin{ingredientsblock}

\begin{malts}
\malt{Two-row}{\lbtokg{10.5}}
\malt{Munich}{\lbtokg{1.5}}
\malt{Carapils}{\lbtokg{0.5}}
\malt{Caramel / Crystal 60 L}{\lbtokg{0.25}}
\end{malts}

\begin{hops}
\hop{\hopcitra}{12~\%}{\fwh}{\oztog{0.5}}
\hop{\hopcitra}{12~\%}{20~min}{\oztog{1}}
\hop{\hopcitra}{12~\%}{10~min}{\oztog{1}}
\hop{\hopcitra}{12~\%}{5~min}{\oztog{1}}
\hop{\hopcitra}{12~\%}{\whirl{}{15~min}}{\oztog{1.5}}
\hop{\hopcitra}{12~\%}{\dryh{}{7~days}}{\oztog{3}}
\end{hops}

\begin{yeasts}
\yeast{White Labs WLP002}
\end{yeasts}

\end{ingredientsblock}

\end{recipie}

% -----------------------------------------------------------------------------
\begin{recipie}{Indeed Brewing Co. Day Tripper Pale Ale}
% -----------------------------------------------------------------------------

\begin{aboutblock}
Day Tripper is a West Coast-style pale ale out of Minneapolis. Indeed Brewing
describes Day Tripper as having a heady, dank, citrus-laced aroma supported by a
complex and subtly sweet malt backbone.
\end{aboutblock}

\specifications{\styleamericanpaleale}{\galtol{5}}{1.052}{}{5.4~\%}{}{}{90~min}{}

\begin{methodandtiming}
 
\begin{mashsteps}
\mashstep{\ftoc{150}}{60~min}
\end{mashsteps}

\begin{fermentationsteps}
\fermentationstep{\ftoc{67}}{}
\end{fermentationsteps}

\begin{directions}
Age one week in secondary. Keg, or bottle with \oztog{5} priming sugar and condition
for 2 weeks.
\end{directions}

\end{methodandtiming}

\pagebreak

\begin{ingredientsblock}

\begin{malts}
\malt{Briess Pale Ale}{\lbtokg{5.5}}
\malt{Maris Otter}{\lbtokg{3.5}}
\malt{White Wheat}{\lbtokg{0.75}}
\malt{Briess Caramel 20 L}{\lbtokg{0.5}}
\malt{Briess Carapils}{\lbtokg{0.5}}
\malt{Briess Bonlander Munich 10 L}{\lbtokg{0.25}}

\end{malts}

\begin{hops}
\hop{\hopwillamette}{}{\fwh}{\oztog{0.25}}
\hop{\hopcascade}{}{20~min}{\oztog{1.5}}
\hop{\hopcascade}{}{10~min}{\oztog{1.5}}
\hop{\hopcolumbus}{}{10~min}{\oztog{0.5}}
\hop{\hopsummit}{}{10~min}{\oztog{0.5}}
\hop{\hopcascade}{}{\foh{}}{\oztog{1.5}}
\hop{\hopcolumbus}{}{\foh{}}{\oztog{1.5}}
\hop{\hopsummit}{}{\foh{}}{\oztog{1}}
\hop{\hopcascade}{}{\dryh{}{3~days}}{\oztog{1.5}}
\hop{\hopcolumbus}{}{\dryh{}{3~days}}{\oztog{1.5}}
\hop{\hopsummit}{}{\dryh{}{3~days}}{\oztog{1}}

\end{hops}

\begin{yeasts}
\yeast{Fermentis SafAle US-05 / Wyeast 1272}
\end{yeasts}

\end{ingredientsblock}

\end{recipie}

% -----------------------------------------------------------------------------
\begin{recipie}{Lynnwood Brewing Drop Bear APA}
% -----------------------------------------------------------------------------

\begin{aboutblock}
Drop Bear APA from Lynnwood Grill \& Brewing (Raleigh, N.C.) is brewed with Mosaic,
El Dorado, and Galaxy hops that are nicely balanced by a mix of two-row, Munich,
and caramel malts.
\end{aboutblock}

\specifications{\styleamericanpaleale}{\galtol{5.5}}{}{}{5.5~\%}{}{\srmtoebc{5.3}}{60~min}{}

\begin{methodandtiming}
 
\begin{mashsteps}
\mashstep{\ftoc{153}}{60~min}
\end{mashsteps}

\begin{fermentationsteps}
\fermentationstep{\ftoc{67}}{}
\end{fermentationsteps}

\begin{directions}
Water treatment: sulfate to chloride at 1:1.
\end{directions}

\end{methodandtiming}

\pagebreak

\begin{ingredientsblock}

\begin{malts}
\malt{Great Western Premium Tow-row}{\lbtokg{11}}
\malt{Briess Bonlander Munich 10 L}{\oztokg{6}}
\malt{Briess Caramel 20 L}{\oztokg{4}}
\malt{Briess Caramel 40 L}{\oztokg{4}}
\end{malts}

\begin{hops}
\hop{\hopmosaic}{12.25~\%}{15~min}{\oztog{0.5}}
\hop{\hopeldorado}{15~\%}{10~min}{\oztog{0.7}}
\hop{\hopgalaxy}{14~\%}{5~min}{\oztog{0.7}}
\hop{\hopeldorado}{15~\%}{\whirl{}{}}{\oztog{0.8}}
\hop{\hopgalaxy}{14~\%}{\whirl{}{}}{\oztog{0.8}}
\hop{\hopmosaic}{12.25~\%}{\whirl{}{}}{\oztog{0.8}}
\hop{\hopgalaxy}{14~\%}{\dryh{}{}}{\oztog{3}}
\hop{\hopeldorado}{15~\%}{\dryh{}{}}{\oztog{1.5}}
\hop{\hopmosaic}{12.25~\%}{\dryh{}{}}{\oztog{1.5}}
\end{hops}

\begin{yeasts}
\yeast{Fermentis SafAle S04 / Wyeast 1056 / White Labs WLP001}
\end{yeasts}

\end{ingredientsblock}

\end{recipie}

% -----------------------------------------------------------------------------
\begin{recipie}{Maine Beer Co. Peeper Ale}
% -----------------------------------------------------------------------------

\begin{aboutblock}
Peeper Pale Ale was the first recipe the Kleban brothers perfected when they decided
to open their brewery in 2009. Maine Beer Co. (Freeport, Maine) describes Peeper as dry,
clean, and well balanced, with a generous dose of American hops.
\end{aboutblock}

\specifications{\styleamericanpaleale}{\galtol{5}}{1.047}{1.007}{}{}{}{60~min}{}

\begin{methodandtiming}
 
\begin{mashsteps}
\mashstep{\ftoc{150}}{}
\end{mashsteps}

\begin{fermentationsteps}
\fermentationstep{\ftoc{68}}{}
\end{fermentationsteps}

\end{methodandtiming}

\pagebreak

\begin{ingredientsblock}

\begin{malts}
\malt{Pale}{\lbtokg{8}}
\malt{Vienna}{\lbtokg{0.5}}
\malt{Red Wheat}{\oztokg{6}}
\malt{Carapils}{\oztokg{6}}
\end{malts}

\begin{hops}
\hop{\hopmagnum}{10~\%}{60~min}{\oztog{0.25}}
\hop{\hopamarillo}{9.2~\%}{10~min}{\oztog{0.15}}
\hop{\hopcascade}{5.5~\%}{10~min}{\oztog{0.2}}
\hop{\hopcentennial}{10~\%}{10~min}{\oztog{0.2}}
\hop{\hopamarillo}{9.2~\%}{\whirl{}{}}{\oztog{1.5}}
\hop{\hopcascade}{5.5~\%}{\whirl{}{}}{\oztog{2}}
\hop{\hopcentennial}{10~\%}{\whirl{}{}}{\oztog{2}}
\hop{\hopamarillo}{9.2~\%}{\dryh{}{}}{\oztog{2.5}}
\hop{\hopcentennial}{10~\%}{\dryh{}{}}{\oztog{2.5}}
\end{hops}

\begin{yeasts}
\yeast{Wyeast 1056}
\end{yeasts}

\end{ingredientsblock}

\end{recipie}

% -----------------------------------------------------------------------------
\begin{recipie}{Maplewood Brewing Co. Charlatan American Pale Ale}
% -----------------------------------------------------------------------------

\begin{aboutblock}
Charlatan is hopped with Simcoe, Citra, and Centennial, and this pale ale is packed
with tropical flavors like mango, passionfruit, and grapefruit according to Chicago's
Maplewood Brewery.
\end{aboutblock}

\specifications{\styleamericanpaleale}{\galtol{5}}{1.059}{1.012}{6.1~\%}{32}{\srmtoebc{5.1}}{90~min}{2.55}

\begin{methodandtiming}
 
\begin{mashsteps}
\mashstep{\ftoc{152}}{45~min}
\end{mashsteps}

\begin{directions}
Water adjustment: 115--125 ppm calcium, 65--75 ppm chloride, 145--155 ppm sulfate.
Target mash pH 5.3. At flameout, use a large spoon or stirrer and create a whirlpool
to cool the wort for a few minutes. The closer you can get to \ftoc{190} the better,
but anything under \ftoc{200} will work. Add the whirlpool hops and let rest 10
minutes, then whirlpool again for a few minutes to create a nice compact trub / hop
pile. When the fermentation is about 1~°P from final gryvity, add the dry hops.
After 3--7 days after final gravity, crash and rack off of the hops.
\end{directions}

\end{methodandtiming}

\pagebreak

\begin{ingredientsblock}

\begin{malts}
\malt{Rahr Standard Two-row}{\lbtokg{8}}
\malt{Weyermann Munich I}{\lbtokg{0.75}}
\malt{Crisp Dextrin}{\lbtokg{0.33}}
\malt{Crisp Cara}{\lbtokg{0.33}}
\malt{Rahr White Wheat}{\lbtokg{0.33}}
\end{malts}

\begin{hops}
\hop{\hopcolumbus}{16.5~\%}{90~min}{\oztog{0.1}}
\hop{\hopsimcoe}{13~\%}{15~min}{\oztog{0.2}}
\hop{\hopcitra}{12.5~\%}{10~min}{\oztog{0.1}}
\hop{\hopcentennial}{10~\%}{10~min}{\oztog{0.1}}
\hop{\hopsimcoe}{13~\%}{10~min}{\oztog{0.2}}
\hop{Whirlfloc Tablet}{}{10~min}{1}
\hop{Yeast Nutrient}{}{10~min}{1}
\hop{\hopcitra}{12.5~\%}{\whirl{}{10~min}}{\oztog{0.33}}
\hop{\hopcentennial}{10~\%}{\whirl{}{10~min}}{\oztog{0.33}}
\hop{\hopsimcoe}{13~\%}{\whirl{}{10~min}}{\oztog{0.165}}
\hop{\hopcitra}{12.5~\%}{\dryh{}{3~days}}{\oztog{1.25}}
\hop{\hopcentennial}{10~\%}{\dryh{}{3~days}}{\oztog{0.75}}
\hop{\hopsimcoe}{13~\%}{\dryh{}{3~days}}{\oztog{0.75}}
\end{hops}

\begin{yeasts}
\yeast{Wyeast 1318}
\end{yeasts}

\end{ingredientsblock}

\end{recipie}