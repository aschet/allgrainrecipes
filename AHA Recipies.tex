\documentclass[10pt,oneside]{scrbook}

\usepackage{scrlayer-scrpage}
\usepackage[utf8]{inputenc}
\usepackage[american]{babel}
\usepackage[pass]{geometry}
\usepackage[regular,condensed,sfdefault]{roboto}
\usepackage[T1]{fontenc}
\usepackage{fp}
\usepackage{xcolor}
\usepackage{mdframed}
\usepackage{colortbl}
\usepackage{graphicx}
\usepackage{tabu}
\usepackage{booktabs}
\usepackage[version=4]{mhchem}
\usepackage{xifthen}
\usepackage{imakeidx}
\usepackage[hidelinks,pdfencoding=auto,pdftex,
  pdfauthor={Thomas Ascher},
  pdfusetitle,
  pdfkeywords={beer, brewing, recipes}]{hyperref}
\usepackage{multicol}
\usepackage{microtype}

\KOMAoptions{toc=flat,toc=listof}

\addtokomafont{part}{\Huge}
\addtokomafont{chapter}{\Huge}
\addtokomafont{section}{\Huge}

\definecolor{softgray}{HTML}{F1F1F1}
\definecolor{hardgray}{HTML}{6A6A6A}
\chead{}
\setlength{\parindent}{0mm} 
\RedeclareSectionCommand[tocnumwidth=0.85cm]{part}

\newmdenv[linewidth=3pt,
linecolor=black,
backgroundcolor=softgray,
fontcolor=hardgray,
rightline=false,
leftline=false,
bottomline=false,
skipabove=1mm,
skipbelow=1.5mm]{recipeframe}

\newcommand{\stylecategory}[1]{\part{#1}}
\newcommand{\stylesection}[1]{\chapter{#1} \clearpage}

\newenvironment{recipe}[2][]{\begin{multicols}{2}[\section*{#2}]\ifthenelse{\equal{#1}{}}{\index{#2}}{\index{#1@#2}}}{\end{multicols} \newpage}

\newcommand{\recipebreak}{\vfill\null \columnbreak}

\newenvironment{recipeblock}[1]
{\uppercase{\textbf{#1}} \begin{recipeframe}}{\end{recipeframe}}

\newenvironment{aboutblock}
{\begin{recipeblock}{About This Recipe}}{\end{recipeblock}}

\newenvironment{ingredientsblock}
{\begin{recipeblock}{Ingredients}}{\end{recipeblock}}

\newcommand{\recipesection}[2]{
\begin{minipage}[t][1.2cm]{\textwidth}
	\begin{minipage}[c][1.2cm][c]{0.95cm}
		\includegraphics[width=0.8cm]{#1}
	\end{minipage}
	\begin{minipage}[c][1.2cm][c]{5cm}
		\large{\textcolor{black}{\uppercase{\textbf{#2}}}}
	\end{minipage}
	\hfill
\end{minipage}
}

\newcommand{\kgtokg}[1]{#1 ~kg}
\newcommand{\gtog}[1]{#1 ~g}
\newcommand{\ppmtopptm}[1]{#1 ~ppm}
\newcommand{\mltoml}[1]{#1 ~ml}
\newcommand{\ltol}[1]{#1 ~l}
\newcommand{\ptop}[1]{#1 ~°P}
\newcommand{\ctoc}[1]{#1 ~°C}
\newcommand{\ibutoibu}[1]{\FPeval\amountg{round((#1):0)}\FPprint\amountg ~IBU}
\newcommand{\psitobar}[1]{\FPeval\amountg{round((#1 * 0.0689476):1)}\FPprint\amountg ~bar}
\newcommand{\ftoc}[1]{\FPeval\amountg{round(((#1 - 32.0) / 1.8):0)}\FPprint\amountg ~°C}
\newcommand{\gpgaltogpl}[1]{\FPeval\amountg{round((#1 * 0.264172):1)}\FPprint\amountg ~g/l}
\newcommand{\ozpgaltogpl}[1]{\FPeval\amountg{round((#1 * 7.48915):1)}\FPprint\amountg ~g/l}
\newcommand{\tsptog}[1]{\FPeval\amountg{round((#1 * 4.92892):0)}\FPprint\amountg ~g}
\newcommand{\tbsptog}[1]{\FPeval\amountg{round((#1 * 15.0):0)}\FPprint\amountg ~g}
\newcommand{\oztog}[1]{\FPeval\amountg{round((#1 * 28.34952):0)}\FPprint\amountg ~g}
\newcommand{\oztokg}[1]{\FPeval\amountg{round((#1 * 0.02834952):2)}\FPprint\amountg ~kg}
\newcommand{\lbtokg}[1]{\FPeval\amountg{round((#1 * 0.4535924):2)}\FPprint\amountg ~kg}
\newcommand{\galtol}[1]{\FPeval\amountg{round((#1 * 3.78541):1)}\FPprint\amountg ~l}
\newcommand{\qttol}[1]{\FPeval\amountg{round((#1 * 0.94635295):1)}\FPprint\amountg ~l}
\newcommand{\oztoml}[1]{\FPeval\amountg{round((#1 * 29.5735):0)}\FPprint\amountg ~ml}
\newcommand{\cuptoml}[1]{\FPeval\amountg{round((#1 * 240.0):0)}\FPprint\amountg ~ml}
\newcommand{\tsptoml}[1]{\FPeval\amountg{round((#1 * 4.92892):0)}\FPprint\amountg ~ml}
\newcommand{\flakecuptog}[1]{\FPeval\amountg{round((#1 * 75.0):0)}\FPprint\amountg ~g}
\newcommand{\sugarcuptog}[1]{\FPeval\amountg{round((#1 * 225.0):0)}\FPprint\amountg ~g}
\newcommand{\srmtoebc}[1]{\FPeval\amountg{round((#1 * 1.97):0)}\FPprint\amountg ~EBC}
\newcommand{\ltoebc}[1]{\FPeval\amountg{round(((#1 * 2.65) - 1.2):0)}\FPprint\amountg ~EBC}
\newcommand{\ebctoebc}[1]{\FPeval\amountg{round((#1):0)}\FPprint\amountg ~EBC}
\newcommand{\sgtop}[1]{\FPeval\amountg{round((135.997 * pow(3,#1) - 630.272 * pow(2,#1) + 1111.14 * #1 - 616.868):1)}\FPprint\amountg ~°P}
\newcommand{\voltogl}[1]{\FPeval\amountg{round((#1 * 1.96):1)}\FPprint\amountg ~g/l}
\newcommand{\lboztokg}[2]{\FPeval\amountg{round((#1 * 0.4535924 + #2 * 28.34952 / 1000.0):2)}\FPprint\amountg ~kg}
\newcommand{\cacotohco}[1]{\FPeval\amountg{round((61.0 * #1 / 50.0):0)}\FPprint\amountg}

\newcommand{\specitem}[1]{\textcolor{black}{\uppercase{\textbf{#1}}}}

\newenvironment{malts} {\recipesection{images/malt.pdf}{Malt / Mash Additions}
\begin{tabu} to \textwidth {Xr}
\textbf{Name} & \textbf{Amount} \\ \midrule}{\end{tabu}}
\newcommand{\malt}[2]{#1 & #2 \\ \midrule}

\newenvironment{hops} {\recipesection{images/hop.pdf}{Hops / Boil Additions}
\begin{tabu} to \textwidth {Xcr}
\textbf{Name} & \textbf{Addition} & \textbf{Amount} \\ \midrule}{\end{tabu}
{\centering\small{FWP = first wort hopping, FO = flameout, WP = whirlpool, DHBT =
biotransformation, DH = dry hopping} \par}}
\newcommand{\hop}[4]{#1 \ifthenelse{\isempty{#2}}{}{(#2)} & \ifthenelse{\isempty{#3}}{--}{#3} & #4 \\ \midrule}

\newenvironment{yeastsx} {\recipesection{images/yeast.pdf}{Yeast}
\begin{tabu} to \textwidth {Xr}
\textbf{Name} & \textbf{Addition} \\ \midrule}{\end{tabu}}
\newcommand{\yeastx}[2]{#1 & #2 \\ \midrule}

\newenvironment{yeasts} {\recipesection{images/yeast.pdf}{Yeast}
\begin{tabu} to \textwidth {X}}{\end{tabu}}
\newcommand{\yeast}[1]{#1 \\ \midrule}

\newcommand{\singleyeast}[1]{
\begin{yeasts}
\yeast{#1}
\end{yeasts}
}

\newenvironment{twists} {\recipesection{images/star.pdf}{Fermentation Additons}
\begin{tabu} to \textwidth {Xcr}
\textbf{Name} & \textbf{Addition} & \textbf{Amount} \\ \midrule}{\end{tabu}}
\newcommand{\twist}[3]{#1 & \ifthenelse{\isempty{#2}}{--}{#2} & #3 \\ \midrule}

\newenvironment{mashsteps} {\recipesection{images/mash.pdf}{Mash}
\begin{tabu} to \textwidth {Xr}}{\end{tabu}}
\newcommand{\mashstep}[2]{#1 & #2 \\ \midrule}
\newcommand{\mashdecoctthin}[1]{\mashstep{}{Thin decoction #1}}
\newcommand{\mashdecoctthick}[1]{\mashstep{}{Thick decoction #1}}
\newcommand{\mashdecoctboil}[1]{\mashstep{\ftoc{212}}{Boil #1}}
\newcommand{\mashdecoctreturn}[2]{\mashstep{#1}{Return to main mash\ifthenelse{\isempty{#2}}{}{; #2}}}

\newenvironment{fermentationsteps} {\recipesection{images/ferment.pdf}{Fermentation} 
\begin{tabu} to \textwidth {Xr}}{\end{tabu}}
\newcommand{\fermentationstep}[2]{#1 & #2 \\ \midrule}

\newenvironment{directions} {\recipesection{images/bulp.pdf}{Directions}}{}

\newenvironment{methodandtiming} {\begin{recipeblock}{Method / Timings}}{\end{recipeblock}}

\newenvironment{yeastinfos}[1] {\begin{recipeblock}{#1} 
\begin{tabu} to \textwidth {lX}}{\end{tabu}
\end{recipeblock}}
\newcommand{\yeastinfo}[2]{#1 & #2 \\ \midrule}

\newenvironment{maltinfos}[1] {\begin{recipeblock}{#1} 
\begin{tabu} to \textwidth {Xrr}
\textbf{Name} & \textbf{Color Min} & \textbf{Color Max} \\ \midrule
}{\end{tabu}
\end{recipeblock}}
\newcommand{\maltinfo}[3]{#1 & #2 & #3 \\ \midrule}

\newcommand{\dryhbt}[2]{DHBT\ifthenelse{\isempty{#1}}{}{\textsubscript{#1}}\ifthenelse{\isempty{#2}}{}{ (#2)}}
\newcommand{\dryh}[2]{DH\ifthenelse{\isempty{#1}}{}{\textsubscript{#1}}\ifthenelse{\isempty{#2}}{}{ (#2)}}
\newcommand{\whirl}[2]{WP\ifthenelse{\isempty{#1}}{}{\textsubscript{#1}}\ifthenelse{\isempty{#2}}{}{ (#2)}}
\newcommand{\fwh}{FWH}
\newcommand{\foh}[1]{FO\ifthenelse{\isempty{#1}}{}{ (#1)}}

\newcommand{\hopahtanum}{Ahtanum}
\newcommand{\hopamarillo}{Amarillo}
\newcommand{\hopapollo}{Apollo}
\newcommand{\hoparamis}{Aramis}
\newcommand{\hopaurora}{Aurora}
\newcommand{\hopazacca}{Azacca}
\newcommand{\hopbramlingcross}{Bramling Cross}
\newcommand{\hopbravo}{Bravo}
\newcommand{\hopcalypso}{Calypso}
\newcommand{\hopcascade}{Cascade}
\newcommand{\hopceleia}{Celeia}
\newcommand{\hopcentennial}{Centennial}
\newcommand{\hopchallenger}{Challenger}
\newcommand{\hopchinook}{Chinook}
\newcommand{\hopcitra}{Citra}
\newcommand{\hopcolumbus}{Columbus}
\newcommand{\hopcomet}{Comet}
\newcommand{\hopcrystal}{Crystal}
\newcommand{\hopeastkentgolding}{East Kent Golding}
\newcommand{\hopekuanot}{Ekuanot}
\newcommand{\hopeldorado}{El Dorado}
\newcommand{\hopeureka}{Eureka}
\newcommand{\hopfalconersflight}{Falconer's Flight}
\newcommand{\hopfuggle}{Fuggle}
\newcommand{\hopgalaxy}{Galaxy}
\newcommand{\hopgalena}{Galena}
\newcommand{\hopglacier}{Glacier}
\newcommand{\hopgolding}{Golding}
\newcommand{\hophallertaublanc}{Hallertau Blanc}
\newcommand{\hophallertaumittelfruh}{Hallertau Mittelfrüh}
\newcommand{\hophallertautradition}{Hallertau Tradition}
\newcommand{\hopherkules}{Herkules}
\newcommand{\hophersbrucker}{Hersbrucker}
\newcommand{\hophorizon}{Horizon}
\newcommand{\hophullmelon}{Hüll Melon}
\newcommand{\hopidahoseven}{Idaho 7}
\newcommand{\hopkohatu}{Kohatu}
\newcommand{\hoplegacy}{Legacy}
\newcommand{\hoplemondrop}{Lemondrop}
\newcommand{\hopliberty}{Liberty}
\newcommand{\hoploral}{Loral}
\newcommand{\hoplubelska}{Lubelska}
\newcommand{\hopmagnum}{Magnum}
\newcommand{\hopmandarinabavaria}{Mandarina Bavaria}
\newcommand{\hopmarynka}{Marynka}
\newcommand{\hopmatueka}{Matueka}
\newcommand{\hopmillennium}{Millennium}
\newcommand{\hopmosaic}{Mosaic}
\newcommand{\hopmotueka}{Motueka}
\newcommand{\hopmthood}{Mt. Hood}
\newcommand{\hopnelsonsauvin}{Nelson Sauvin}
\newcommand{\hopnorthdown}{Northdown}
\newcommand{\hopnorthernbrewer}{Northern Brewer}
\newcommand{\hopnugget}{Nugget}
\newcommand{\hopopal}{Opal}
\newcommand{\hoppacifica}{Pacifica}
\newcommand{\hoppacificgem}{Pacific Gem}
\newcommand{\hoppacificjade}{Pacific Jade}
\newcommand{\hoppalisade}{Palisade}
\newcommand{\hopperle}{Perle}
\newcommand{\hoppolaris}{Polaris}
\newcommand{\hoprakau}{Rakau}
\newcommand{\hopsaaz}{Saaz}
\newcommand{\hopsantiam}{Santiam}
\newcommand{\hopsimcoe}{Simcoe}
\newcommand{\hopsorachiace}{Sorachi Ace}
\newcommand{\hopspalt}{Spalt}
\newcommand{\hopspaltselect}{Spalt Select}
\newcommand{\hopsterling}{Sterling}
\newcommand{\hopstrisselspalt}{Strisselspalt}
\newcommand{\hopstyriangolding}{Styrian Golding}
\newcommand{\hopsummit}{Summit}
\newcommand{\hoptarget}{Target}
\newcommand{\hoptettnang}{Tettnang}
\newcommand{\hopvanguard}{Vanguard}
\newcommand{\hopvicsecret}{Vic Secret}
\newcommand{\hopwakatu}{Wakatu}
\newcommand{\hopwarrior}{Warrior}
\newcommand{\hopwillamette}{Willamette}
\newcommand{\hopzythos}{Zythos}

% Light Lager
\newcommand{\styleamericanlightlager}{American Light Lager}
\newcommand{\styleamericanlager}{American Lager}
\newcommand{\styleinternationalpalelager}{International Pale Lager}
\newcommand{\stylemunichhelles}{Munich Helles}
\newcommand{\stylefestbier}{Festbier}
\newcommand{\stylegermanhellesexportbier}{German Helles Exportbier}

% Pilsner
\newcommand{\stylegermanleichtbier}{German Leichtbier}
\newcommand{\styleczechpalelager}{Czech Pale Lager}
\newcommand{\stylegermanpils}{German Pils}
\newcommand{\styleczechpremiumpalelager}{Czech Premium Pale Lager}

% European Amber Lager
\newcommand{\styleviennalager}{Vienna Lager}
\newcommand{\styleczechamberlager}{Czech Amber Lager}
\newcommand{\stylemarzen}{Märzen}
\newcommand{\stylekellerbier}{Kellerbier}

% Bock
\newcommand{\stylehellesbock}{Helles Bock}
\newcommand{\styledunklesbock}{Dunkles Bock}
\newcommand{\styledoppelbock}{Doppelbock}
\newcommand{\styleeisbock}{Eisbock}

% Dark Lager
\newcommand{\styleinternationaldarklager}{International Dark Lager}
\newcommand{\stylemunichdunkel}{Munich Dunkel}
\newcommand{\styleczechdarklager}{Czech Dark Lager}
\newcommand{\styleschwarzbier}{Schwarzbier}

% Light Hybrid Beer
\newcommand{\stylecreamale}{Cream Ale}
\newcommand{\styleblondeale}{Blonde Ale}
\newcommand{\stylekolsch}{Kölsch}
\newcommand{\styleamericanwheat}{American Wheat}

% Amber Hybrid Beer
\newcommand{\styleinternationalamberlager}{International Amber Lager}
\newcommand{\stylecaliforniacommon}{California Common}
\newcommand{\stylealtbier}{Altbier}

% English Pale Ale
\newcommand{\styleordinarybitter}{Ordinary Bitter}
\newcommand{\stylebestbitter}{Best Bitter}
\newcommand{\stylestrongbitter}{Strong Bitter}
\newcommand{\stylebritishgoldenale}{British Golden Ale}
\newcommand{\styleaustraliansparklingale}{Australian Sparkling Ale}

% Scottish & Irish Ale
\newcommand{\stylescottishlight}{Scottish Light}
\newcommand{\stylescottishheavy}{Scottish Heavy}
\newcommand{\stylescottishexport}{Scottish Export}
\newcommand{\styleirishredale}{Irish Red Ale}
\newcommand{\styleweeheavy}{Wee Heavy}

% American Pale Ale
\newcommand{\styleamericanpaleale}{American Pale Ale}

% Other American Ale
\newcommand{\styleamericanamberale}{American Amber Ale}
\newcommand{\styleamericanbrownale}{American Brown Ale}

% English Brown Ale
\newcommand{\styledarkmild}{Dark Mild}
\newcommand{\stylebritishbrownale}{British Brown Ale}

% Porter
\newcommand{\styleenglishporter}{English Porter}
\newcommand{\styleamericanporter}{American Porter}
\newcommand{\stylebalticporter}{Baltic Porter}

% Stout
\newcommand{\styleoatmealstout}{Oatmeal Stout}
\newcommand{\stylesweetstout}{Sweet Stout}

% Strong Stout
\newcommand{\styleamericanstout}{American Stout}
\newcommand{\styleimperialstout}{Imperial Stout}

% American IPA
\newcommand{\styleamericanipa}{American IPA}

% India Pale Ale
\newcommand{\englishipa}{English IPA}
\newcommand{\styleryeipa}{Specialty IPA: Rye IPA}

% German Wheat Rye Beer
\newcommand{\styleweissbier}{Weissbier}
\newcommand{\styledunklesweissbier}{Dunkles Weissbier}
\newcommand{\styleweizenbock}{Weizenbock}

% Belgian & French Al
\newcommand{\stylewitbier}{Witbier}
\newcommand{\stylebelgianpaleale}{Belgian Pale Ale}
\newcommand{\stylesaison}{Saison}

% Belgian Strong Ale
\newcommand{\stylebelgianblondale}{Belgian Blond Ale}
\newcommand{\stylebelgiandubbel}{Belgian Dubbel}
\newcommand{\stylebelgiantripel}{Belgian Tripel}
\newcommand{\stylebelgiangoldenstrongale}{Belgian Golden Strong Ale}
\newcommand{\stylebelgiandarkstrongale}{Belgian Dark Strong Ale}

% Strong Ale
\newcommand{\styleoldale}{Old Ale}
\newcommand{\styleenglishbarleywine}{English Barleywine}
\newcommand{\styleamericanbarleywine}{American Barleywine}

% Fruit Beer
\newcommand{\stylefruitbeer}{Fruit Beer}

% Spice / Herb / Vegetable Beer
\newcommand{\stylewinterseasonalbeer}{Winter Seasonal Beer}

% Smoke-Flavored & Wood-Aged Beer
\newcommand{\stylerauchbier}{Rauchbier}

% Specialty Beer
\newcommand{\stylealternativesugarbeer}{Alternative Sugar Beer}

\newcommand{\sourcehomebrewchallenge}{Source: The Homebrew Challenge.}
\newcommand{\sourceaha}{Source: American Homebrewers Association.}
\newcommand{\sourcezymurgy}[1]{Source: Zymurgy #1.}
\newcommand{\waterprofile}[6]{\ce{Ca} #1~ppm, \ce{Mg} #2~ppm, \ce{Na} #3~ppm, \ce{SO4} #4~ppm, \ce{Cl} #5~ppm\ifthenelse{\isempty{#6}}{}{, \ce{HCO3} #6~ppm}}

\newcommand{\specifications}[9]{
\begin{recipeblock}{Specifications}
\begin{tabu} to \textwidth {Xr}
\specitem{Style} & #1 \\ \midrule
\specitem{Volume} & #2 \\ \midrule
\specitem{Original Gravity} & \ifthenelse{\isempty{#3}}{--}{\sgtop{#3} / #3} \\ \midrule
\specitem{Final Gravity} & \ifthenelse{\isempty{#4}}{--}{\sgtop{#4} / #4} \\ \midrule
\specitem{ABV} & \ifthenelse{\isempty{#5}}{--}{#5} \\ \midrule
\specitem{Bitterness} & \ifthenelse{\isempty{#6}}{--}{\ibutoibu{#6}} \\ \midrule
\specitem{Color} & \ifthenelse{\isempty{#7}}{--}{#7} \\ \midrule
\specitem{Boil Time} & \ifthenelse{\isempty{#8}}{--}{#8} \\ \midrule
\specitem{Carbonation} & \ifthenelse{\isempty{#9}}{--}{\voltogl{#9} / #9~vol} \\ \midrule
\end{tabu}
\end{recipeblock}
}

\newgeometry{left=2.5cm,right=2.5cm,top=2.5cm,bottom=3cm}

\RedeclareSectionCommand[beforeskip=0mm,afterindent=false,afterskip=3mm]{chapter}
\RedeclareSectionCommand[afterindent=false,afterskip=0.5mm]{section}
\renewcommand*{\partpagestyle}{empty} 


\begin{document}

\title{American Homebrewers Association Recipe Collection}
\date{\today}
\maketitle

\frontmatter

\tableofcontents

\mainmatter
\twocolumn

\part{\styleamericanale}

\chapter*{Boiler Brewing Co. Bobber's Big Red Rye Pale Ale}

\begin{aboutblock}
This beer from Boiler Brewing Company was created as a collaboration with Lincoln, Nebraska,
homebrewer Bob Catherall of the world famous Lincoln Lagers Homebrew Club. The spicy rye notes
blend so perfectly with the new school Mosaic hops that it's hard to tell where the rye ends
and the hops begin. Boiler was lucky enough to win a Gold Medal at the U.S. Open Beer Championship
with the very first batch of this beer.
\end{aboutblock}

\specifications{\styleamericanale}{\galtol{5.5}}{1.058}{1.015}{5.6~\%}{40}{\srmtoebc{12.5}}{}

\begin{methodandtiming}
 
\begin{mashsteps}
\mashstep{\ftoc{155}}{60~min}
\end{mashsteps}

\begin{fermentationsteps}
\fermentationstep{\ftoc{66}}{}
\end{fermentationsteps}

\begin{directions}
Dry hop with the pellets for 3 days, then cold crash before kegging. The whole leaf Mosaic
hops go in the keg in a mesh bag for as long as the keg lasts (it won't last long). Before
racking to the keg, add the whole leaf hops, then pump up to 15 PSI with \ce{CO2}, then vent,
repeating several times to remove as much \ce{O2} as possible before racking.
\end{directions}

\end{methodandtiming}

\pagebreak

\begin{ingredientsblock}

\begin{malts}
\malt{Simpsons Golden Promise}{\lbtokg{8}}
\malt{Flaked Rye}{\lbtokg{2}}
\malt{Melanoidin}{\lbtokg{1}}
\malt{Pale Wheat}{\lbtokg{1}}
\malt{Weyermann CARAMUNICH II}{\oztokg{8}}
\malt{Dingemans Special B}{\oztokg{8}}
\malt{Rice Hulls}{\oztokg{8}}
\end{malts}

\begin{hops}
\hop{\hopcolumbus}{11.4~\%}{\fwh}{\oztog{1}}
\hop{Whirlfloc Tablet}{}{15~min}{1}
\hop{\hopmosaic}{11.8~\%}{\dryh{}{3~days}}{\oztog{5}}
\hop{\hopmosaic ~Cones}{8.9~\%}{\dryh{}{3~days}}{\oztog{3}}
\end{hops}

\begin{yeasts}
\yeast{Lallemand BRY-97 American West Coast Ale}
\end{yeasts}

\end{ingredientsblock}

\chapter*{Boise Brewing Co. Down Down Extra Pale Ale}

\begin{aboutblock}
Inspired by a community garden, Boise Brewing has been sharing their small batch beer
with the city of Tress since 2014. Centennial hops are the main event in this pale ale -- give it a whirl!
\end{aboutblock}

\specifications{\styleamericanale}{\galtol{10}}{1.045}{}{}{40}{}{60~min}

\begin{methodandtiming}
 
\begin{mashsteps}
\mashstep{\ftoc{149}}{60~min}
\end{mashsteps}

\end{methodandtiming}

\pagebreak

\begin{ingredientsblock}

\begin{malts}
\malt{Pale}{\lbtokg{14.5}}
\malt{Crystal 15 L}{\lbtokg{2.5}}
\malt{Flaked Barley}{\oztokg{9}}
\end{malts}

\begin{hops}
\hop{\hopcentennial}{10~\%}{60~min}{\oztog{1.0}}
\hop{\hopcentennial}{10~\%}{30~min}{\oztog{0.5}}
\hop{\hopcentennial}{10~\%}{10~min}{\oztog{0.75}}
\hop{\hopcentennial}{10~\%}{\foh}{\oztog{1.0}}
\hop{\hopcascade}{}{\dryh{}{}}{\oztog{1.0}}
\hop{\hopcentennial}{}{\dryh{}{}}{\oztog{1.0}}
\end{hops}

\begin{yeasts}
\yeast{White Labs WLP001}
\end{yeasts}

\end{ingredientsblock}

\part{\styleamericanamberale}

\chapter*{Iron Monk Brewing Co. Velvet Antler Amber}

\begin{aboutblock}
Iron Monk Brewing's American amber ale is very malt-forward and smooth. They hear from
fans all the time that this is their all-time favorite beer, and we think you’ll agree
once you brew some yourself.
\end{aboutblock}

\specifications{\styleamericanamberale}{\galtol{5}}{1.045}{}{4.6~\%}{14}{}{60~min}

\begin{methodandtiming}
 
\begin{mashsteps}
\mashstep{\ftoc{154}}{60~min}
\mashstep{\ftoc{170}}{10~min}
\end{mashsteps}

\begin{fermentationsteps}
\fermentationstep{\ftoc{68}}{}
\end{fermentationsteps}

\end{methodandtiming}

\pagebreak

\begin{ingredientsblock}

\begin{malts}
\malt{Two-row}{\lbtokg{7.0}}
\malt{Caramel 60 L}{\lbtokg{2.0}}
\malt{Munich Light}{\lbtokg{0.5}}
\malt{Red}{\lbtokg{0.5}}
\malt{Flaked Wheat}{\lbtokg{0.5}}
\malt{Flaked Barley}{\lbtokg{0.5}}
\end{malts}

\begin{hops}
\hop{\hopsimcoe}{}{60~min}{\oztog{0.4}}
\hop{\hopcascade}{}{\foh}{\oztog{0.3}}
\end{hops}

\begin{yeasts}
\yeast{Fermentis SafAle S-05}
\end{yeasts}

\end{ingredientsblock}

\part{\styleamericanipa}

\chapter*{Bell's Two Hearted Ale Clone}

\begin{aboutblock}
Bell's Brewery of Kalamazoo, Mich. brews a little beer called Two Hearted Ale. Maybe you've
heard of it? This India pale ale is bursting with hop aromas ranging from pine to grapefruit
thanks to the use of 100 percent Centennial hops. This recipe was created by David Curtis and
Ryan Kramer of Bell's General Store.
\end{aboutblock}

\specifications{\styleamericanipa}{\galtol{5}}{1.063}{1.012}{6.7~\%}{55}{\srmtoebc{10}}{75~min}

\begin{methodandtiming}
 
\begin{mashsteps}
\mashstep{\ftoc{150}}{45~min}
\mashstep{\ftoc{170}}{raise to over 15~min}
\mashstep{\ftoc{170}}{10~min}
\end{mashsteps}

\begin{directions}
Use carbon filtered water, adjust with 4 g gypsum. Ferment warm (ale temperature).
Dry hop one week into fermentation. Allow two hearted clone to stay warm with hops for a week.
Rack beer, crash cool, and cold age for a week.
\end{directions}

\end{methodandtiming}

\pagebreak

\begin{ingredientsblock}

\begin{malts}
\malt{Briess Brewers}{\lbtokg{10}}
\malt{Briess Pale Ale}{\lbtokg{2.83}}
\malt{Briess Caramel 40 L}{\oztokg{8}}
\end{malts}

\begin{hops}
\hop{\hopcentennial}{9.1~\%}{45~min}{\oztog{1.2}}
\hop{\hopcentennial}{9.1~\%}{30~min}{\oztog{1.2}}
\hop{\hopcentennial}{9.1~\%}{\dryh{}{}}{\oztog{3.5}}
\end{hops}

\begin{yeasts}
\yeast{White Labs WLP001 / White Labs WLP051}
\end{yeasts}

\end{ingredientsblock}

\chapter*{Bissel Brothers Brewing The Substance New England IPA}

\begin{aboutblock}
The Substance from Bissel Brothers Brewing Co. flirts with the new world IPA style in a way
that intrigues and compels, adding complexity and not detracting from the beer. It does have
notes of tropical citrus, but it is still first and foremost "dank" with a perceived bitterness
that contributes to an overall balanced experience.
\end{aboutblock}

\specifications{\styleamericanipa}{\galtol{5.16}}{1.061}{1.011}{}{}{}{}

\begin{methodandtiming}
 
\begin{mashsteps}
\mashstep{\ftoc{150}}{}
\end{mashsteps}

\begin{fermentationsteps}
\fermentationstep{\ftoc{68}}{--}
\fermentationstep{\ftoc{71}}{raise to on second day}
\end{fermentationsteps}

\end{methodandtiming}

\pagebreak

\begin{ingredientsblock}

\begin{malts}
\malt{Pale Two-row}{\lbtokg{10}}
\malt{Flaked Wheat}{\lbtokg{1.18}}
\malt{Crystal 20 L}{\oztokg{5.6}}
\malt{Flaked Oats}{\oztokg{3.8}}
\end{malts}

\begin{hops}
\hop{\hopapollo}{}{Start}{\oztog{1}}
\hop{\hopfalconersflight}{}{\foh}{\oztog{1}}
\hop{\hopcentennial}{}{\foh}{\oztog{1}}
\hop{\hopfalconersflight}{}{\dryh{}{}}{\oztog{3}}
\hop{\hopcentennial}{}{\dryh{}{}}{\oztog{1.5}}
\hop{\hopeureka}{}{\dryh{}{}}{\oztog{1}}
\hop{\hopapollo}{}{\dryh{}{}}{\oztog{1}}
\hop{\hopchinook}{}{\dryh{}{}}{\oztog{1}}

\end{hops}

\begin{yeasts}
\yeast{Wyeast 2112 / White Labs WLP810}
\end{yeasts}

\end{ingredientsblock}

\chapter*{Corridor Brewery Wizard Fight American IPA}

\begin{aboutblock}
This flagship beer from Corridor Brewery features a plethora of cool kid hops
including Mosaic, Citra, and El Dorado create a citrus and tropical paradise.
\end{aboutblock}

\specifications{\styleamericanipa}{\galtol{5}}{1.059}{1.010}{6.5~\%}{60}{}{90~min}

\begin{methodandtiming}
 
\begin{mashsteps}
\mashstep{\ftoc{152}}{60~min}
\end{mashsteps}

\begin{fermentationsteps}
\fermentationstep{\ftoc{68}}{}
\end{fermentationsteps}

\begin{directions}
Addition of 0.5 g \ce{CaSO4} and 0.5 g \ce{CaCl2} at mash in. Cold crash one
day after dry hopping.
\end{directions}

\end{methodandtiming}

\pagebreak

\begin{ingredientsblock}

\begin{malts}
\malt{Two-row}{\lbtokg{8}}
\malt{Bonlander Munich}{\oztokg{12}}
\malt{Carapils}{\oztokg{12}}
\malt{Flaked Oats}{\oztokg{4}}
\end{malts}

\begin{hops}
\hop{\hopwarrior}{16~\%}{90~min}{\oztog{0.75}}
\hop{\hopchinook}{13~\%}{15~min}{\oztog{0.5}}
\hop{\hopcitra}{13~\%}{\foh}{\oztog{0.64}}
\hop{\hopeldorado}{15~\%}{\foh}{\oztog{0.64}}
\hop{\hopmosaic}{11~\%}{\foh}{\oztog{0.64}}
\hop{\hopmosaic}{}{\dryh{}{5~days}}{\oztog{0.5}}
\hop{\hopcitra}{}{\dryh{}{5~days}}{\oztog{0.5}}
\hop{\hopeldorado}{}{\dryh{}{5~days}}{\oztog{0.5}}
\end{hops}

\begin{yeasts}
\yeast{Brewing Science Institute A-18}
\end{yeasts}

\end{ingredientsblock}

\chapter*{Deschutes' Fresh Squeezed IPA Clone}

\begin{aboutblock}
The name Fresh Squeezed IPA gives you an idea of what's in store when you brew this clone. First and
foremost, you should drink this beer fresh. This hop-centric IPA has big, piney hop aroma that's full
of fruit and peppery notes. It drips with juicy citrus and grapefruit flavor thanks to the Citra hops,
while the Mosaic hops present soft, fruit flavors like honeydew. A mild malt profile of pale, Munich
and crystal take a back seat to the hops, making this easy to drink IPA.
\end{aboutblock}

\specifications{\styleamericanipa}{\galtol{5.5}}{1.066}{1.014}{7~\%}{60}{\srmtoebc{10}}{90~min}

\begin{methodandtiming}
 
\begin{mashsteps}
\mashstep{\ftoc{150}}{60~min}
\end{mashsteps}

\begin{fermentationsteps}
\fermentationstep{\ftoc{70}}{}
\end{fermentationsteps}

\begin{directions}
To brew this Fresh Squeezed IPA clone, use 1 g/gal gypsum to treat distilled or reverse osmosis water. Drink it as fresh as possible (2--3 weeks after packaging) for maximum late hop character.
\end{directions}

\end{methodandtiming}

\pagebreak

\begin{ingredientsblock}

\begin{malts}
\malt{Pale Two-row}{\lbtokg{11}}
\malt{Munich}{\lbtokg{1.75}}
\malt{Crystal 75 L}{\lbtokg{0.75}}
\end{malts}

\begin{hops}
\hop{\hopnugget}{13~\%}{60~min}{\oztog{0.5}}
\hop{\hopcitra}{12~\%}{15~min}{\oztog{1}}
\hop{\hopmosaic}{12~\%}{15~min}{\oztog{1}}
\hop{Whirlfloc Tablet}{}{10~min}{1}
\hop{\hopcitra}{12~\%}{\whirl{}{10~min}}{\oztog{1}}
\hop{\hopcitra}{12~\%}{\dryh{}{5~days}}{\oztog{1}}
\hop{\hopmosaic}{12~\%}{\dryh{}{5~days}}{\oztog{1}}

\end{hops}

\begin{yeasts}
\yeast{White Labs WLP001}
\end{yeasts}

\end{ingredientsblock}

\chapter*{Fargo Brewing Company Wood Chipper IPA}

\begin{aboutblock}
This classic American IPA from Fargo Brewing Company showcases aromatic and bold
hop flavors. Horizon hops and oats provide a sleek, velvety body and balanced bitterness
while pounds per barrel of Cascade, Centennial, Chinook and Simcoe hops give this IPA
waves of citrus and pine flavors. That's one delicious beer eh? Oh yeah, you betcha!
\end{aboutblock}

\specifications{\styleamericanipa}{\galtol{5}}{1.062}{}{6.5~\%}{59}{\srmtoebc{4.5}}{60~min}

\begin{methodandtiming}
 
\begin{mashsteps}
\mashstep{\ftoc{155}}{}
\end{mashsteps}

\end{methodandtiming}

\pagebreak

\begin{ingredientsblock}

\begin{malts}
\malt{Pale}{\lbtokg{10.5}}
\malt{Weyermann Munich I}{\lbtokg{1}}
\malt{Rice Hulls}{--}
\end{malts}

\begin{hops}
\hop{\hopcentennial}{5.8~\%}{FWH}{\oztog{0.5}}
\hop{\hophorizon}{11.3~\%}{60~min}{\oztog{0.75}}
\hop{Whirlfloc Tablet}{}{15~min}{1}
\hop{\hopcascade}{7.55~\%}{10~min}{\oztog{0.75}}
\hop{\hopchinook}{11.1~\%}{10~min}{\oztog{0.5}}
\hop{Servomyces}{}{10~min}{1}
\hop{\hopsimcoe}{}{\dryh{}{}}{\oztog{1}}
\hop{\hopchinook}{}{\dryh{}{}}{\oztog{1}}
\hop{\hopcentennial}{}{\dryh{}{}}{\oztog{1}}
\hop{\hopcascade}{}{\dryh{}{}}{\oztog{1}}
\end{hops}

\begin{yeasts}
\yeast{American Ale}
\end{yeasts}

\end{ingredientsblock}

\chapter*{Focal Point (Inspired by The Alchemist's Focal Banger)}

\begin{aboutblock}
The Alchemist's Focal Banger recipe is a closely guarded secret, but this American IPA is
inspired by famous hoppy ale. This recipe was formulated by the AHA with a few tips from
John Kimmich, head brewer and owner of the Stow, Vt.-based brewery.
\end{aboutblock}

\specifications{\styleamericanipa}{\galtol{5.5}}{1.064}{1.012}{7~\%}{80}{\srmtoebc{5}}{60~min}

\begin{methodandtiming}

\begin{mashsteps}
\mashstep{\ftoc{150}}{75~min}
\end{mashsteps}

\begin{directions}
If you don't have the equipment to conduct a whirlpool at the end of the boil, simply conduct a
hop stand by steeping the final addition of Mosaic in the hot wort for 10 minutes before chilling.
Add dry hops for 3 days, prior to packaging. 
\end{directions}

\end{methodandtiming}

\pagebreak

\begin{ingredientsblock}

\begin{malts}
\malt{Pale}{\lbtokg{9}}
\malt{Pilsner}{\lbtokg{4.8}}
\end{malts}

\begin{hops}
\hop{Hop Extract}{}{60~min}{6~ml}
\hop{\hopmosaic}{12.25~\%}{10~min}{\oztog{1}}
\hop{\hopmosaic}{12.25~\%}{5~min}{\oztog{1}}
\hop{\hopmosaic}{12.25~\%}{\whirl{}{}}{\oztog{1}}
\hop{\hopcitra}{12.25~\%}{\dryh{}{3~days}}{\oztog{4}}
\end{hops}

\begin{yeasts}
\yeast{White Labs WLP095 / Omega Yeast OLY-052 / GigaYeast GY054 / The Yeast Bay WLP4000 / Imperial Yeast A04}
\end{yeasts}

\end{ingredientsblock}

\chapter*{Good Word Brewing Never Sleep New England IPA}

\begin{aboutblock}
This is one of Good Word Brewing's most popular beers, described as "juicy" without being overly
sweet thanks to Vic Secret and Citra hops. Pilsner malt dominates alongside English pale malt and
oats for a little more body and mouthfeel.
\end{aboutblock}

\specifications{\styleamericanipa}{\galtol{5}}{1.065}{1.013}{7~\%}{45}{\srmtoebc{3.8}}{90~min}

\begin{methodandtiming}
 
\begin{mashsteps}
\mashstep{\ftoc{146}}{15~min}
\mashstep{\ftoc{156}}{30~min}
\mashstep{\ftoc{168}}{10~min}
\end{mashsteps}

\begin{fermentationsteps}
\fermentationstep{\ftoc{68}}{2~days}
\fermentationstep{\ftoc{70}}{2~days}
\fermentationstep{\ftoc{72}}{--}
\end{fermentationsteps}

\begin{directions}
After boil, add whirlpool hops once wort is below \ftoc{180} to prevent isomerization of hops.
Dry hop for 3 days when final gravity is within 0.5--1.0 °P. Complete a diacetyl rest before
cold crashing. Do this by taking a \oztog{2} sample that can be capped. Place sample in
\ftoc{140} water for 20 minutes. Allow sample to come down to room temperature and test for
diacetyl by smell and taste. If still present wait another 24--48 hours and retest.
Only cold crash after the sample has passed the test. Crash at \ftoc{32} for 4--6 days.
\end{directions}

\end{methodandtiming}

\pagebreak

\begin{ingredientsblock}

\begin{malts}
\malt{Pilsner}{\lbtokg{6.4}}
\malt{Pale}{\lbtokg{3.25}}
\malt{Oat}{\lbtokg{1.5}}
\end{malts}

\begin{hops}
\hop{Dextrose}{}{}{\oztog{14}}
\hop{\hopvicsecret}{}{10~min}{\oztog{1}}
\hop{\hopvicsecret}{}{5~min}{\oztog{1.5}}
\hop{\hopcitra}{}{5~min}{\oztog{1.5}}
\hop{\hopcitra}{}{\whirl{}{}}{\oztog{5}}
\hop{\hopvicsecret}{}{\whirl{}{}}{\oztog{5}}
\hop{\hopvicsecret}{}{\dryh{}{3~days}}{\oztog{8}}
\hop{\hopcitra}{}{\dryh{}{3~days}}{\oztog{8}}
\end{hops}

\begin{yeasts}
\yeast{Unspecified}
\end{yeasts}

\end{ingredientsblock}

\chapter*{Great South Bay Brewery Massive IPA}

\begin{aboutblock}
Let's just say this Massive IPA from New York's Great South Bay Brewery doesn't skimp on
the hops! With a hefty dose of flameout hops, Massive IPA is packed full of American hop
flavor and aroma.
\end{aboutblock}

\specifications{\styleamericanipa}{\galtol{5}}{1.065}{1.011}{6.7~\%}{55}{\srmtoebc{10}}{60~min}

\begin{methodandtiming}
 
\begin{mashsteps}
\mashstep{\ftoc{151}}{}
\end{mashsteps}

\begin{directions}
Adjust water with \tsptog{1} \ce{CaSO4}. Dry hop at high krausen and carbonate to 2.5 volumes \ce{CO2}.
\end{directions}

\end{methodandtiming}

\pagebreak

\begin{ingredientsblock}

\begin{malts}
\malt{Two-row}{\lbtokg{10.5}}
\malt{Weyermann CARAAMBER}{\lbtokg{0.75}}
\malt{Crystal 30 L}{\lbtokg{0.5}}
\malt{Carapils}{\lbtokg{0.6}}
\malt{Pre-gelatinized Flaked Oats}{\lbtokg{0.5}}
\end{malts}

\begin{hops}
\hop{\hopchinook}{}{60~min}{\oztog{0.75}}
\hop{\hopcentennial}{}{1~min}{\oztog{0.5}}
\hop{\hopcascade}{}{1~min}{\oztog{1}}
\hop{\hopsimcoe}{}{1~min}{\oztog{0.5}}
\hop{\hopcascade}{}{\foh}{\oztog{0.25}}
\hop{\hopcentennial}{}{\foh}{\oztog{0.25}}
\hop{\hopsimcoe}{}{\foh}{\oztog{0.5}}
\hop{\hopcascade}{}{\dryh{}{}}{\oztog{0.5}}
\hop{\hopsimcoe}{}{\dryh{}{}}{\oztog{0.5}}
\end{hops}

\begin{yeasts}
\yeast{American Ale}
\end{yeasts}

\end{ingredientsblock}

\chapter*{Green Flash West Coast IPA}

\begin{aboutblock}
The plethora of hops creates a layered drinking experience of bitterness and hop-forward flavor
and aroma. The additions of Simcoe and Centennial give off tropical grapefruit and piney notes
respectively, while the Cascade hops add some floral quality to compliment the spicy citrus notes
of the Columbus and Amarillo hops.
With an assertive flavor and grapefruit bitterness held up by crystal malt, it's a beer of extremes -- overdoing things just because they wanted to. And you should to!
\end{aboutblock}

\specifications{\styleamericanipa}{\galtol{5.5}}{1.075}{1.018}{7.48~\%}{90}{\srmtoebc{7}}{90~min}

\begin{methodandtiming}
 
\begin{mashsteps}
\mashstep{\ftoc{152}}{60~min}
\mashstep{\ftoc{165}}{10~min}
\end{mashsteps}

\begin{fermentationsteps}
\fermentationstep{\ftoc{70}}{}
\end{fermentationsteps}

\begin{directions}
To brew Green Flash West Coast IPA, the water profile should be similar to Burton-Upon-Trent's,
but with roughly half the mineral content. When fermentation is complete, dry hop in primary
fermenter for seven days. Drink it fresh for maximum late hop character.
\end{directions}

\end{methodandtiming}

\pagebreak

\begin{ingredientsblock}

\begin{malts}
\malt{Pale Two-row}{\lbtokg{12.5}}
\malt{Bairds Carastan}{\lbtokg{1.25}}
\malt{Briess Dextrin}{\lbtokg{1.25}}
\end{malts}

\begin{hops}
\hop{\hopsimcoe}{13~\%}{90~min}{\oztog{0.5}}
\hop{\hopcolumbus}{14~\%}{90~min}{\oztog{0.25}}
\hop{\hopsimcoe}{13~\%}{60~min}{\oztog{0.25}}
\hop{\hopcolumbus}{14~\%}{60~min}{\oztog{0.25}}
\hop{\hopsimcoe}{13~\%}{30~min}{\oztog{0.25}}
\hop{\hopcolumbus}{14~\%}{30~min}{\oztog{0.25}}
\hop{\hopsimcoe}{13~\%}{15~min}{\oztog{0.75}}
\hop{\hopcolumbus}{14~\%}{15~min}{\oztog{0.75}}
\hop{Whirlfloc Tablet}{}{10~min}{1}
\hop{\hopsimcoe}{13~\%}{\whirl{}{10~min}}{\oztog{0.5}}
\hop{\hopcolumbus}{14~\%}{\whirl{}{10~min}}{\oztog{0.5}}
\hop{\hopcascade}{5.75~\%}{\whirl{}{10~min}}{\oztog{1}}
\hop{\hopsimcoe}{13~\%}{\dryh{}{7~days}}{\oztog{0.5}}
\hop{\hopcascade}{5.75~\%}{\dryh{}{7~days}}{\oztog{0.5}}
\hop{\hopamarillo}{10~\%}{\dryh{}{7~days}}{\oztog{0.5}}
\hop{\hopcentennial}{10.5~\%}{\dryh{}{7~days}}{\oztog{0.5}}
\end{hops}

\begin{yeasts}
\yeast{White Labs WLP001}
\end{yeasts}

\end{ingredientsblock}

\chapter*{Lazy Magnolia Brewery Southern Hops'pitality IPA}

\begin{aboutblock}
This session IPA from Lazy Magnolia Brewing in Kiln, Miss. is meant for sharing with
family and friends. It's described as having a bold citrus burst on the front end, with
hints of tropical fruits such as grapefruit, orange, and mango in the finish.
\end{aboutblock}

\specifications{\styleamericanipa}{\galtol{5.5}}{1.054}{1.010}{5.8~\%}{}{}{90~min}

\begin{methodandtiming}
 
\begin{mashsteps}
\mashstep{\ftoc{152}}{60~min}
\end{mashsteps}

\begin{fermentationsteps}
\fermentationstep{\ftoc{65}}{}
\end{fermentationsteps}

\begin{directions}
Use water with Calcium Chloride and food-grade acid added to increase hardness and reduce alkalinity
if necessary. Our water is extremely alkaline with zero hardness and calcium. Unless your water is
extremely soft, mineral additions may not be required. Different water profiles will result in
different hop flavors, so have fun with it.
\end{directions}

\end{methodandtiming}

\pagebreak

\begin{ingredientsblock}

\begin{malts}
\malt{Two-row}{\lbtokg{10}}
\malt{Carapils}{\lbtokg{0.5}}
\malt{Crystal 40 L}{\lbtokg{0.5}}
\end{malts}

\begin{hops}
\hop{\hopnugget}{}{60~min}{\oztog{0.25}}
\hop{\hopcentennial}{}{10~min}{\oztog{0.5}}
\hop{\hopcitra}{}{10~min}{\oztog{0.5}}
\hop{\hopsimcoe}{}{10~min}{\oztog{1}}
\hop{Whirlfloc Tablet}{}{10~min}{1}
\hop{\hopcentennial}{}{5~min}{\oztog{0.6}}
\hop{\hopcitra}{}{5~min}{\oztog{0.6}}
\hop{\hopsimcoe}{}{5~min}{\oztog{0.6}}
\hop{\hopcitra}{}{\whirl{}{}}{\oztog{0.75}}
\hop{\hopsummit}{}{\whirl{}{}}{\oztog{0.6}}
\hop{\hopsimcoe}{}{\whirl{}{}}{\oztog{0.6}}
\hop{\hopfalconersflight}{}{\whirl{}{}}{\oztog{0.6}}
\hop{\hopcitra}{}{\dryh{}{}}{\oztog{1.5}}
\end{hops}

\begin{yeasts}
\yeast{Fermentis SafAle US-05 / White Labs WLP001}
\end{yeasts}

\end{ingredientsblock}

\chapter*{Lupulin Brewing Straight Hash Homie IPA}

\begin{aboutblock}
This appropriately named IPA from Lupulin Brewing is made with pure lupulin powder,
paying homage to the brewery's name. The bursting tropical flavors and soft bitterness
may fool you, but no pellet or hop touched this beer. Brew your own and see what you think!
\end{aboutblock}

\specifications{\styleamericanipa}{\galtol{6}}{1.076}{1.018}{7.75~\%}{60}{\srmtoebc{5.5}}{60~min}

\begin{methodandtiming}
 
\begin{mashsteps}
\mashstep{\ftoc{152}}{}
\end{mashsteps}

\begin{fermentationsteps}
\fermentationstep{\ftoc{66}}{--}
\fermentationstep{\ftoc{72}}{raise to over 1 week}
\end{fermentationsteps}

\end{methodandtiming}

\pagebreak

\begin{ingredientsblock}

\begin{malts}
\malt{Two-row}{\lbtokg{10.5}}
\malt{Caramalt}{\lbtokg{1}}
\malt{Flaked Rye}{\lbtokg{1}}
\malt{Vienna}{\lbtokg{1}}
\end{malts}

\begin{hops}
\hop{Dextrose}{}{60~min}{\oztog{12}}
\hop{\hopcitra ~Hash}{}{\whirl{}{}}{\oztog{2}}
\hop{\hopsimcoe ~Hash}{}{\whirl{}{}}{\oztog{2}}
\hop{\hopcitra ~Hash}{}{\dryh{}{3~days}}{\oztog{3}}
\hop{\hopmosaic ~Hash}{}{\dryh{}{3~days}}{\oztog{2}}
\end{hops}

\begin{yeasts}
\yeast{Omega Yeast OYL-052}
\end{yeasts}

\end{ingredientsblock}

\chapter*{Odell IPA}

\begin{aboutblock}
Odell Brewing Co. of Fort Collins, Colorado makes one tasty IPA, and the 2007 Great American
Beer Festival judges agreed when they awarded Odell IPA with a gold medal in the
"American-style IPA" category.
\end{aboutblock}

\specifications{\styleamericanipa}{\galtol{5.5}}{1.067}{}{}{47}{\srmtoebc{7}}{90~min}

\begin{methodandtiming}
 
\begin{mashsteps}
\mashstep{\ftoc{154}}{60~min}
\mashstep{\ftoc{168}}{10~min}
\end{mashsteps}

\begin{fermentationsteps}
\fermentationstep{\ftoc{68}}{--}
\fermentationstep{\ftoc{60}}{1 week}
\end{fermentationsteps}

\begin{directions}
Use a hopback at runoff for \oztog{1} Simcoe and Chinook additions, or steep whole flowers
at flameout for 10 minutes.
\end{directions}

\end{methodandtiming}

\pagebreak

\begin{ingredientsblock}

\begin{malts}
\malt{Gambrinus ESB Pale}{\lbtokg{6}}
\malt{Pale}{\lbtokg{5}}
\malt{Vienna}{\lbtokg{2}}
\malt{Thomas Fawcett Caramalt}{\oztokg{10}}
\malt{Weyermann CARAFOAM}{\oztokg{8}}
\end{malts}

\begin{hops}
\hop{\hophorizon}{13~\%}{90~min}{\oztog{0.75}}
\hop{\hopsimcoe}{13~\%}{90~min}{\oztog{0.5}}
\hop{\hopcolumbus}{15~\%}{Hopback}{\oztog{1}}
\hop{\hopchinook}{13~\%}{Hopback}{\oztog{1}}
\hop{\hopsimcoe}{13~\%}{\dryh{}{}}{\oztog{0.5}}
\hop{\hophorizon}{13~\%}{\dryh{}{}}{\oztog{0.5}}
\hop{\hopamarillo}{13~\%}{\dryh{}{}}{\oztog{0.5}}
\hop{\hopcentennial}{13~\%}{\dryh{}{}}{\oztog{0.5}}
\end{hops}

\begin{yeasts}
\yeast{Nottingham Ale}
\end{yeasts}

\end{ingredientsblock}

\chapter*{Perfect Plain Brewing Co. Holy Spin American IPA}

\begin{aboutblock}
Dubbed the Holy Spin, the third spin of a vinyl record is known to be when the tunes
are at their best. This IPA recipe from Perfect Plain Brewing Co. was the third turn
of their brewhouse and is dry-hopped, abundantly so, with citra hops to the tune
of 3.3 lb/bbl!
\end{aboutblock}

\specifications{\styleamericanipa}{\galtol{5}}{1.061}{1.010}{6.5~\%}{43}{\srmtoebc{5}}{90~min}

\begin{methodandtiming}
 
\begin{mashsteps}
\mashstep{\ftoc{150}}{}
\end{mashsteps}

\begin{directions}
Targeting a mash pH between 5.2 and 5.4. Chill wort below \ftoc{170} or below (if possible)
before adding whirlpool hops for 15 minutes. Dry hop with \oztog{2} of Citra hops 48 hours
after the start of primary fermentation, being careful to prevent as much oxygen pickup as
possible. Once final gravity is reached, add the remaining \oztog{5} of Citra dry hops for
5 days. Crash and carbonate to 2.5 volumes of \ce{CO2}.
\end{directions}

\end{methodandtiming}

\pagebreak

\begin{ingredientsblock}

\begin{malts}
\malt{Pale Two-row}{\lbtokg{8.5}}
\malt{Unmalted White Wheat}{\lbtokg{2.1}}
\malt{Flaked Oats}{\oztokg{10}}
\end{malts}

\begin{hops}
\hop{\hopnugget}{12.5~\%}{60~min}{\oztog{1}}
\hop{\hopcitra}{}{\whirl{}{}}{\oztog{2}}
\hop{\hopcitra}{}{\dryhbt{}{}}{\oztog{2}}
\hop{\hopcitra}{}{\dryh{}{5~days}}{\oztog{5}}
\end{hops}

\begin{yeasts}
\yeast{Fermentis SafAle US-05}
\end{yeasts}

\end{ingredientsblock}

\chapter*{Providence Brewing Company Battlecow Galacticose New England IPA}

\begin{aboutblock}
This deceptively strong beer from Providence Brewing Company is brewed with Two-row pale
malt, Cara\-foam, rolled oats, milk sugar, and intensely double dry-hopped with Citra and
Mosaic giving it a decidedly dank aroma. Bursting with mango, orange and pineapple flavors,
this is a juicy milkshake New England IPA that'll have you begging for another sip.
\end{aboutblock}

\specifications{\styleamericanipa}{\galtol{5.5}}{1.071}{1.019}{8.1~\%}{70}{\srmtoebc{3.6}}{60~min}

\begin{methodandtiming}
 
\begin{mashsteps}
\mashstep{\ftoc{155}}{60~min}
\end{mashsteps}

\begin{fermentationsteps}
\fermentationstep{\ftoc{70}}{}
\end{fermentationsteps}

\begin{directions}
Turn off heat and let temperature drop to \ftoc{198}, then add the whirlpool hops and
whirlpool for 30 minutes. Two days after pitching the yeast, add the first dry hop
additions. After active fermentation has stopped, typically 5 days after pitching the
yeast, add the second round of dry hops. A day later, burp your fermenter (if possible)
with a 30 second burst of \ce{CO2} to rouse the hops. Add 2 oz Citra and 2 oz Mosaic pellet hops 8-10 days later. Just before kegging your beer, bag and add last dry hop additions to the keg or bottling bucket.
\end{directions}

\end{methodandtiming}

\pagebreak

\begin{ingredientsblock}

\begin{malts}
\malt{Two-row}{\lbtokg{10}}
\malt{Flaked Oats}{\lbtokg{2}}
\malt{Weyermann CARAFOAM}{\lbtokg{1.5}}
\malt{White Wheat}{\lbtokg{1.5}}
\end{malts}

\begin{hops}
\hop{\hopcolumbus ~Cones}{14~\%}{FWH}{\oztog{2}}
\hop{\hopcolumbus}{14~\%}{60~min}{\oztog{2}}
\hop{Lactose}{}{10~min}{\lbtokg{2}}
\hop{\hopcitra}{12~\%}{\whirl{}{30~min}}{\oztog{2}}
\hop{\hopmosaic}{12.25~\%}{\whirl{}{30~min}}{\oztog{2}}
\hop{\hopcitra}{12~\%}{\dryhbt{}{}}{\oztog{1}}
\hop{\hopmosaic}{12.25~\%}{\dryhbt{}{}}{\oztog{1}}
\hop{\hopcitra}{12~\%}{\dryh{1}{4~days}}{\oztog{1}}
\hop{\hopmosaic}{12.25~\%}{\dryh{1}{4~days}}{\oztog{1}}
\hop{\hopcitra}{12~\%}{\dryh{2}{10~days}}{\oztog{2}}
\hop{\hopmosaic}{12.25~\%}{\dryh{2}{10~days}}{\oztog{2}}
\hop{\hopcitra ~Cryo}{26~\%}{\dryh{3}{kegging}}{\oztog{1}}
\hop{\hopmosaic ~Cryo}{26~\%}{\dryh{3}{kegging}}{\oztog{1}}
\end{hops}

\begin{yeasts}
\yeast{The Yeast Bay WLP4042}
\end{yeasts}

\end{ingredientsblock}

\chapter*{Red Door Brewing Company New England IPA}

\begin{aboutblock}
This juicy and hazy India pale ale from Red Door Brewing Co. features an intense tropical
fruit and floral nose. This is a perfect choice for warm weather.
\end{aboutblock}

\specifications{\styleamericanipa}{\galtol{5}}{1.066}{1.014}{7.1~\%}{69}{\srmtoebc{4.3}}{60~min}

\begin{methodandtiming}
 
\begin{mashsteps}
\mashstep{\ftoc{154}}{60~min}
\end{mashsteps}

\begin{fermentationsteps}
\fermentationstep{\ftoc{68}}{}
\end{fermentationsteps}

\begin{directions}
Mash with a girst ratio of 2.5 qt/lb. Add the first dry hops and recirculate. After 2 days,
add the second dry hops and recirculate. Cold crash the next day to \ftoc{33}. Leave cold
crashed for 5 days.
\end{directions}

\end{methodandtiming}

\pagebreak

\begin{ingredientsblock}

\begin{malts}
\malt{Two-row}{\lbtokg{7.45}}
\malt{Malted White Wheat}{\lbtokg{3.77}}
\malt{Flaked Oats}{\lbtokg{2.18}}
\malt{Calcium Chloride}{\oztog{0.32}}
\malt{Phosphoric Acid (85\%)}{2.21~ml}
\end{malts}

\begin{hops}
\hop{Calcium Sulfate}{}{}{\oztog{0.32}}
\hop{\hopcitra}{13.7~\%}{\fwh}{\oztog{0.2}}
\hop{\hopmosaic}{10.7~\%}{\fwh}{\oztog{0.2}}
\hop{\hopeldorado}{15~\%}{\fwh}{\oztog{0.2}}
\hop{Zinc}{}{15~min}{16~ml}
\hop{\hopcitra}{13.7~\%}{\whirl{}{15~min}}{\oztog{1.3}}
\hop{\hopmosaic}{10.7~\%}{\whirl{}{15~min}}{\oztog{1.3}}
\hop{\hopeldorado}{15~\%}{\whirl{}{15~min}}{\oztog{1.3}}
\hop{\hopcitra}{13.7~\%}{\dryh{1}{2~days}}{\oztog{1.3}}
\hop{\hopmosaic}{10.7~\%}{\dryh{1}{2~days}}{\oztog{1.3}}
\hop{\hopeldorado}{15~\%}{\dryh{1}{2~days}}{\oztog{1.3}}
\hop{\hopcitra}{13.7~\%}{\dryh{2}{5~days}}{\oztog{1.3}}
\hop{\hopmosaic}{10.7~\%}{\dryh{2}{5~days}}{\oztog{1.3}}
\hop{\hopeldorado}{15~\%}{\dryh{2}{5~days}}{\oztog{1.3}}
\end{hops}

\begin{yeasts}
\yeast{Wyeast 1318}
\end{yeasts}

\end{ingredientsblock}

\chapter*{Russian River Blind Pig IPA Clone}

\begin{aboutblock}
Vinnie Cilurzo says about the groundbreaking Blind Pig IPA: "Blind Pig IPA was first brewed
in Temecula, Calif. at my first brewery, Blind Pig Brewing Company, in 1994. This version
was 92 bittering units and had very little malt with a very forward hop character. In December
1996, I left the brewery and my former business partner continued the brewery for a few years.
After the original brewery closed and I was at Russian River Brewing Company, we were able to
trademark the name and start making Blind Pig IPA again. The recipe has changed, in that we have
added a couple of new hop varieties that were not in existence when the brewery in Temecula was 
open."
\end{aboutblock}

\specifications{\styleamericanipa}{\galtol{5}}{1.057}{1.013}{6.1~\%}{62}{}{90~min}

\begin{methodandtiming}
 
\begin{mashsteps}
\mashstep{\ftoc{153}}{60~min}
\end{mashsteps}

\begin{fermentationsteps}
\fermentationstep{\ftoc{68}}{}
\end{fermentationsteps}

\end{methodandtiming}

\pagebreak

\begin{ingredientsblock}

\begin{malts}
\malt{Pale Two-row}{\lbtokg{9.8}}
\malt{Crystal 40 L}{\oztokg{6.5}}
\malt{Dextrin}{\oztokg{5}}
\end{malts}

\begin{hops}
\hop{\hopcolumbus}{16~\%}{90~min}{\oztog{0.25}}
\hop{\hopchinook}{13~\%}{90~min}{\oztog{0.5}}
\hop{\hopamarillo}{7.5~\%}{30~min}{\oztog{0.5}}
\hop{\hopcascade}{5.75~\%}{\foh}{\oztog{0.5}}
\hop{\hopamarillo}{7.5~\%}{\foh}{\oztog{0.5}}
\hop{\hopcentennial}{10.5~\%}{\foh}{\oztog{0.5}}
\hop{\hopsimcoe}{10.5~\%}{\foh}{\oztog{0.5}}
\hop{\hopcascade}{5.75~\%}{\dryh{}{10~days}}{\oztog{0.5}}
\hop{\hopamarillo}{7.5~\%}{\dryh{}{10~days}}{\oztog{0.5}}
\hop{\hopcolumbus}{16~\%}{\dryh{}{10~days}}{\oztog{0.5}}
\end{hops}

\begin{yeasts}
\yeast{White Labs WLP001 / Wyeast 1056}
\end{yeasts}

\end{ingredientsblock}

\chapter*{Spice Trade Brewing Sun Temple IPA}

\begin{aboutblock}
Recipe courtesy Jeff Tyler, Spice Trade Brewing Co., Arvada, Colo. Tyler says:
Don't use any cold-side clarifying agents! Haze means there are hop polyphenols in
solution, which promote magical, juicy, fruit-forward flavor!
\end{aboutblock}

\specifications{\styleamericanipa}{\galtol{5.5}}{1.064}{1.010}{7.1~\%}{75}{\srmtoebc{7}}{60~min}

\begin{methodandtiming}
 
\begin{mashsteps}
\mashstep{\ftoc{148}}{60~min}
\end{mashsteps}

\begin{fermentationsteps}
\fermentationstep{\ftoc{68}}{}
\end{fermentationsteps}

\begin{directions}
Minerals are important for this style. If you dabble in chemistry, shoot for 100 ppm
chloride and 200 ppm sulfate. Specific salt additions depend on the base water profile
you start with. After flameout, add whirlpool hops and stir wort for 30 minutes to
create a whirlpool and precipitate out the trub. Bottle or keg with 2.6 volumes (5.2 g/L)
of \ce{CO2}.
\end{directions}

\end{methodandtiming}

\pagebreak

\begin{ingredientsblock}

\begin{malts}
\malt{Pale}{\lbtokg{10.63}}
\malt{Weyermann CARAAMER}{\oztokg{6}}
\malt{Dingemans Special B}{\oztokg{3}}
\end{malts}

\begin{hops}
\hop{\hopmagnum}{12.3~\%}{60~min}{\oztog{0.6}}
\hop{\hopeldorado}{9~\%}{20~min}{\oztog{1.9}}
\hop{Yeast Nutrient}{}{15~min}{\tsptog{0.25}}
\hop{\hopcitra}{14.1~\%}{10~min}{\oztog{0.8}}
\hop{Whirlfloc Tablet}{}{10~min}{1}
\hop{Dextrose}{}{10~min}{\lbtokg{1}}
\hop{\hopeldorado}{9~\%}{FO}{\oztog{0.8}}
\hop{\hopsimcoe}{12.3~\%}{FO}{\oztog{0.3}}
\hop{\hopeldorado}{9~\%}{\whirl{}{30~min}}{\oztog{0.75}}
\hop{\hopsimcoe}{12.3~\%}{\whirl{}{30~min}}{\oztog{1.25}}
\hop{\hopeldorado}{9~\%}{\dryh{1}{4~days}}{\oztog{1.2}}
\hop{\hopsimcoe}{12.3~\%}{\dryh{1}{4~days}}{\oztog{0.7}}
\hop{\hopcitra}{14.1~\%}{\dryh{1}{4~days}}{\oztog{0.7}}
\hop{\hopeldorado}{9~\%}{\dryh{2}{3~days}}{\oztog{1.2}}
\hop{\hopsimcoe}{12.3~\%}{\dryh{2}{3~days}}{\oztog{0.7}}
\hop{\hopcitra}{14.1~\%}{\dryh{2}{3~days}}{\oztog{0.7}}
\hop{\hopeldorado}{9~\%}{\dryh{3}{}}{\oztog{1.2}}
\hop{\hopsimcoe}{12.3~\%}{\dryh{3}{}}{\oztog{0.7}}
\hop{\hopcitra}{14.1~\%}{\dryh{3}{}}{\oztog{0.8}}
\end{hops}

\begin{yeasts}
\yeast{GigaYeast GY054 / Inland Island Yeast Lab. INIS-003 / The Yeast Bay WLP 4000}
\end{yeasts}

\end{ingredientsblock}

\chapter*{Uinta Brewing Co. Hop Nosh Tangerine}

\begin{aboutblock}
Hop Nosh Tangerine is a play on Uinta Brewing’s (Salt Lake City, Utah) flagship IPA. This
homebrew recipe features an aromatic menagerie of tropical hops and tangerine zest and wraps
up with a crisp, bitter finish.
\end{aboutblock}

\specifications{\styleamericanipa}{\galtol{5}}{1.065}{}{6.7~\%}{}{}{60~min}

\begin{methodandtiming}
 
\begin{mashsteps}
\mashstep{\ftoc{154}}{}
\end{mashsteps}

\begin{fermentationsteps}
\fermentationstep{\ftoc{66}}{}
\end{fermentationsteps}

\end{methodandtiming}

\pagebreak

\begin{ingredientsblock}

\begin{malts}
\malt{Pale Two-row}{\lbtokg{11}}
\malt{Munich}{\lbtokg{1}}
\malt{Crystal 40 L}{\lbtokg{0.5}}
\end{malts}

\begin{hops}
\hop{\hopchinook}{}{60~min}{\oztog{1}}
\hop{\hopcascade}{}{30~min}{\oztog{1.5}}
\hop{\hopbravo}{}{5~min}{\oztog{1}}
\hop{\hopcascade}{}{5~min}{\oztog{1}}
\hop{Tangerine Concentrate}{}{1~min}{1 L}
\hop{\hopbravo}{}{\whirl{}{}}{\oztog{0.5}}
\hop{\hopcitra}{}{\whirl{}{}}{\oztog{0.5}}
\hop{\hopgalaxy}{}{\whirl{}{}}{\oztog{1}}
\hop{\hopgalaxy}{}{\dryh{}{}}{\oztog{1}}
\hop{\hopcitra}{}{\dryh{}{}}{\oztog{0.75}}
\hop{\hopchinook}{}{\dryh{}{}}{\oztog{0.75}}
\end{hops}

\begin{yeasts}
\yeast{Wyeast 1007}
\end{yeasts}

\end{ingredientsblock}

\chapter*{Von Ebert Brewing Sabrage Brut IPA}

\begin{aboutblock}
Sabrage is the technique for opening a champagne bottle with a saber, a fitting name for
this beer recipe from Von Ebert Brewing. This recipe is dry and effervescent, with minimal
bitterness and hop character that's dominated by grapefruit-heavy Citra and a touch of
resinous Chinook. You don't need a ceremonial occasion to drink this beer!
\end{aboutblock}

\specifications{\styleamericanipa}{\galtol{5.5}}{1.050}{}{6.6~\%}{10}{}{90~min}

\begin{methodandtiming}
 
\begin{mashsteps}
\mashstep{\ftoc{142}}{60~min}
\mashstep{\ftoc{168}}{--}
\end{mashsteps}

\begin{fermentationsteps}
\fermentationstep{\ftoc{65}}{}
\end{fermentationsteps}

\begin{directions}
Force carbonate or bottle condition to 3.0 volumes of \ce{CO2}. For a special version of
this recipe, try incorporating a grape varietal into the fermentation.
\end{directions}

\end{methodandtiming}

\pagebreak

\begin{ingredientsblock}

\begin{malts}
\malt{Pilsner}{\lbtokg{8}}
\malt{Flaked Rice}{\lbtokg{2}}
\end{malts}

\begin{hops}
\hop{\hopcitra}{}{Mash}{\oztog{2}}
\hop{\hopcitra}{}{\fwh}{\oztog{2}}
\hop{\hopcitra}{}{\whirl{}{}}{\oztog{2}}
\hop{\hopcitra}{}{\dryh{}{3~days}}{\oztog{6}}
\hop{\hopchinook}{}{\dryh{}{3~days}}{\oztog{2}}
\end{hops}

\begin{yeasts}
\yeast{Neutral Ale}
\end{yeasts}

\end{ingredientsblock}

\chapter*{WeldWerks Brewing Juicy Bits NEIPA}

\begin{aboutblock}
From WeldWerks: "Our version of a New England-style IPA featuring a huge citrus and tropical
fruit character from the Mosaic, Citra, and El Dorado hops, a softer, fluffier mouthfeel from
the lower attenuation, and the characteristic New England hop haze. The end result is a beer
reminiscent of citrus juice with extra pulp, thus the name Juicy Bits."
\end{aboutblock}

\specifications{\styleamericanipa}{\galtol{5}}{1.062}{1.012}{}{45}{\srmtoebc{4.5}}{90~min}

\begin{methodandtiming}
 
\begin{mashsteps}
\mashstep{\ftoc{149}}{45~min}
\end{mashsteps}

\begin{fermentationsteps}
\fermentationstep{\ftoc{67}}{}
\end{fermentationsteps}

\begin{directions}
If desired, use Epsom salt (\ce{MgSO4}) and calcium chloride for water adjustments, adding
half at mash and half at sparge, targeting about 250 ppm chloride and 80 ppm sulfate.
Add the first dry hop addition when the beer has fermented to about 2--3 °P from final gravity,
and then add the last two dry hop additions after terminal gravity has been reached.
\end{directions}

\end{methodandtiming}

\pagebreak

\begin{ingredientsblock}

\begin{malts}
\malt{Pale Two-row}{\lbtokg{4}}
\malt{Pilsner}{\lbtokg{4}}
\malt{Dextrin}{\lbtokg{1}}
\malt{Pale Wheat}{\lbtokg{1}}
\malt{Flaked Oats}{\oztokg{12}}
\malt{Flaked Wheat}{\oztokg{12}}
\malt{Wheat}{\oztokg{12}}
\end{malts}

\begin{hops}
\hop{\hopmagnum}{14~\%}{FWH}{\oztog{0.33}}
\hop{Dextrose}{}{90~min}{\oztog{6}}
\hop{\hopcitra}{12.5~\%}{\whirl{1}{10~min}}{\oztog{0.33}}
\hop{\hopeldorado}{15.7~\%}{\whirl{1}{10~min}}{\oztog{0.33}}
\hop{\hopmosaic}{13.1~\%}{\whirl{1}{10~min}}{\oztog{0.33}}
\hop{\hopcitra}{12.5~\%}{\whirl{2}{10~min}}{\oztog{0.66}}
\hop{\hopeldorado}{15.7~\%}{\whirl{2}{10~min}}{\oztog{0.66}}
\hop{\hopmosaic}{13.1~\%}{\whirl{2}{10~min}}{\oztog{0.66}}
\hop{\hopcitra}{12.5~\%}{\whirl{3}{20~min}}{\oztog{1}}
\hop{\hopeldorado}{15.7~\%}{\whirl{3}{20~min}}{\oztog{1}}
\hop{\hopmosaic}{13.1~\%}{\whirl{3}{20~min}}{\oztog{1}}
\hop{\hopcitra}{12.5~\%}{\dryhbt{}{}}{\oztog{0.5}}
\hop{\hopeldorado}{15.7~\%}{\dryhbt{}{}}{\oztog{0.5}}
\hop{\hopmosaic}{13.1~\%}{\dryhbt{}{}}{\oztog{0.5}}
\hop{\hopcitra}{12.5~\%}{\dryh{1}{3~days}}{\oztog{0.5}}
\hop{\hopeldorado}{15.7~\%}{\dryh{1}{3~days}}{\oztog{0.5}}
\hop{\hopmosaic}{13.1~\%}{\dryh{1}{3~days}}{\oztog{0.5}}
\hop{\hopcitra}{12.5~\%}{\dryh{2}{3~days}}{\oztog{1}}
\hop{\hopeldorado}{15.7~\%}{\dryh{2}{3~days}}{\oztog{1}}
\hop{\hopmosaic}{13.1~\%}{\dryh{2}{3~days}}{\oztog{1}}
\end{hops}

\end{ingredientsblock}

\pagebreak

\begin{ingredientsblock}

\begin{yeasts}
\yeast{Wyeast 1318}
\end{yeasts}

\end{ingredientsblock}


\chapter*{Whetstone Station Brewery Whetstoner Session IPA}

\begin{aboutblock}
This bright and delicious session IPA from Whetstone Craft Beers features Simcoe,
Amarillo and Citra hops. While it's hazy, aromatic and full of flavor, at just 4.5~\% ABV
this crisp beer is perfect for when you've got thirst that needs quenching.
\end{aboutblock}

\specifications{\styleamericanipa}{\galtol{5}}{1.045}{1.010}{4.6~\%}{36}{\srmtoebc{4.8}}{90~min}

\begin{methodandtiming}
 
\begin{mashsteps}
\mashstep{\ftoc{150}}{}
\mashstep{\ftoc{168}}{}
\end{mashsteps}

\begin{fermentationsteps}
\fermentationstep{\ftoc{65}}{}
\end{fermentationsteps}

\begin{directions}
Carbonate to 2.3 volumes of \ce{CO2}.
\end{directions}

\end{methodandtiming}

\pagebreak

\begin{ingredientsblock}

\begin{malts}
\malt{Pale Two-row}{\lbtokg{7}}
\malt{White Wheat}{\lbtokg{2}}
\malt{Crystal 30 L}{\lbtokg{0.5}}
\end{malts}

\begin{hops}
\hop{\hopamarillo}{9.2~\%}{\whirl{}{}}{\oztog{2.25}}
\hop{\hopsimcoe}{13~\%}{\whirl{}{}}{\oztog{1.5}}
\hop{\hopamarillo}{9.2~\%}{\dryh{}{2~days}}{\oztog{0.75}}
\hop{\hopcitra}{13.4~\%}{\dryh{}{2~days}}{\oztog{0.75}}
\hop{\hopsimcoe}{13~\%}{\dryh{}{2~days}}{\oztog{0.75}}
\end{hops}

\begin{yeasts}
\yeast{American Ale}
\end{yeasts}

\end{ingredientsblock}

\chapter*{Zipline Brewing Co. NZ IPA}

\begin{aboutblock}
Zipline's India pale ale is brewed in Lincoln, Neb. and is packed with New Zealand hop
varieties like Pacific Jade, Rakau, and Wakatu, known for their exotic fruity flavors and
aromas.
\end{aboutblock}

\specifications{\styleamericanipa}{\galtol{5.25}}{1.062}{}{}{}{}{60~min}

\begin{methodandtiming}
 
\begin{mashsteps}
\mashstep{\ftoc{152}}{}
\end{mashsteps}

\begin{directions}
Target mash pH is 5.3. Add dry hops after 9 days in primary fermentation.
\end{directions}

\end{methodandtiming}

\pagebreak

\begin{ingredientsblock}

\begin{malts}
\malt{Briess Pilsner}{\lbtokg{9.15}}
\malt{Weyermann Vienna}{\lbtokg{0.9}}
\malt{Weyermann CARAMUNICH II}{\lbtokg{0.65}}
\malt{Briess Victory}{\lbtokg{0.48}}
\malt{Flaked White Wheat}{\lbtokg{0.8}}
\end{malts}

\begin{hops}
\hop{\hoppacificjade}{}{60~min}{11~g}
\hop{\hoppacificgem}{}{30~min}{11~g}
\hop{\hoprakau}{}{30~min}{11~g}
\hop{\hopwakatu}{}{10~min}{10~g}
\hop{\hoprakau}{}{10~min}{11~g}
\hop{\hopmotueka}{}{10~min}{10~g}
\hop{\hoppacifica}{}{10~min}{11~g}
\hop{\hoppacificgem}{}{\whirl{}{}}{14~g}
\hop{\hopwakatu}{}{\whirl{}{}}{14~g}
\hop{\hoprakau}{}{\whirl{}{}}{14~g}
\hop{\hopmotueka}{}{\whirl{}{}}{14~g}
\hop{\hoppacifica}{}{\whirl{}{}}{14~g}
\hop{\hoppacificgem}{}{\dryh{}{}}{14~g}
\hop{\hopwakatu}{}{\dryh{}{}}{14~g}
\hop{\hoprakau}{}{\dryh{}{}}{14~g}
\hop{\hopmotueka}{}{\dryh{}{}}{14~g}
\hop{\hoppacifica}{}{\dryh{}{}}{14~g}
\end{hops}

\begin{yeasts}
\yeast{IPA}
\end{yeasts}

\end{ingredientsblock}

\backmatter

\part{\styleamericanlager}

\chapter*{Parleaux Beer Lab Lemony Sippit Lager}

\begin{aboutblock}
This hopped-up take on a classic American lager from Parleaux Beer Lab is whirlpooled
with lemongrass and then dry-hopped with Lemondrop hops, lending it a rich, bright aroma
with a smooth lemon candy-like finish.
\end{aboutblock}

\specifications{\styleamericanlager}{\galtol{5}}{1.052}{1.012}{5.2~\%}{21}{\srmtoebc{3.6}}{90~min}

\begin{methodandtiming}
  
\begin{mashsteps}
\mashstep{\ftoc{152}}{}
\end{mashsteps}

\begin{fermentationsteps}
\fermentationstep{\ftoc{50}}{until \sgtop{1.025}}
\fermentationstep{\ftoc{68}}{raise to by 3~°C per day}
\fermentationstep{\ftoc{68}}{48~hours}
\fermentationstep{\ftoc{40}}{3--4~weeks}
\end{fermentationsteps}

\begin{directions}
Fresh lemongrass is always preferred if available. If a heavy lemongrass flavor is preferred
try making a lemongrass tincture, and adding adding to taste at packaging. Simply combine
\oztog{6} of vodka with \oztog{2} of fresh lemongrass in an airtight container and let sit at
room temperature for 2--4 weeks.
\end{directions}

\end{methodandtiming}

\pagebreak

\begin{ingredientsblock}

\begin{malts}
\malt{Weyermann Bohemian Pilsner}{\lbtokg{7.5}}
\malt{Vienna}{\lbtokg{1.5}}
\malt{Flaked Rice}{\lbtokg{1.5}}
\malt{Carapils}{\lbtokg{1}}
\end{malts}

\begin{hops}
\hop{\hoplemondrop}{6~\%}{40~min}{\oztog{1}}
\hop{\hoplemondrop}{6~\%}{\whirl{}{20~min}}{\oztog{2}}
\hop{Lemongrass}{}{\whirl{}{20~min}}{\oztog{2}}
\hop{\hoplemondrop}{6~\%}{\dryh{}{}}{\oztog{2}}
\end{hops}

\begin{yeasts}
\yeast{Wyeast 2278}
\end{yeasts}

\end{ingredientsblock}

\chapter*{Phil's Lager}

\begin{aboutblock}
Philip Blosser of Oregon, OH, member of the Glass City Mashers, won a gold medal in
Category \#1: Light Lager during the 2017 National Homebrew Competition Final Round in
Minneapolis, MN. Blosser’s Light Lager was chosen as the best among 89 entries in the
category.
\end{aboutblock}

\specifications{\styleamericanlager}{\galtol{5}}{1.035}{1.005}{3.9~\%}{}{}{60~min}

\begin{methodandtiming}
  
\begin{mashsteps}
\mashstep{\ftoc{152}}{90~min}
\end{mashsteps}

\begin{fermentationsteps}
\fermentationstep{\ftoc{54}}{15~days}
\fermentationstep{\ftoc{34}}{14~days}
\end{fermentationsteps}

\end{methodandtiming}

\pagebreak

\begin{ingredientsblock}

\begin{malts}
\malt{German Pilsner}{\lbtokg{2.72}}
\malt{Caramel 10 L}{\lbtokg{0.25}}
\malt{Caramel 20 L}{\lbtokg{0.25}}
\end{malts}

\begin{hops}
\hop{\hopcentennial}{8.4~\%}{60~min}{\oztog{0.3}}
\hop{Whirlfloc Tablet}{}{}{1}
\end{hops}

\begin{yeasts}
\yeast{Fermentis SafLager W-34/70}
\end{yeasts}

\end{ingredientsblock}

\part{\styleamericanpaleale}

\chapter*{3 Floyds Alpha King Clone}

\begin{aboutblock}
Alpha King, a bold, citrus-forward American pale ale has been consistently ranked as
one of the best pale ales made in America. If you're a fan of the versatile Centennial
hop, this beer from Munster, Ind.'s 3 Floyds Brewing Co. is for you!
\end{aboutblock}

\specifications{\styleamericanpaleale}{\galtol{5}}{1.059}{1.014}{6.4~\%}{69}{}{60~min}

\begin{methodandtiming}
 
\begin{mashsteps}
\mashstep{\ftoc{154}}{}
\end{mashsteps}

\begin{fermentationsteps}
\fermentationstep{\ftoc{68}}{7~days}
\end{fermentationsteps}

\end{methodandtiming}

\pagebreak

\begin{ingredientsblock}

\begin{malts}
\malt{Pale Two-row}{\lbtokg{10}}
\malt{Simpsons Crystal Medium}{\lbtokg{1}}
\malt{Dingemans Cara 45 L}{\lbtokg{0.5}}
\end{malts}

\begin{hops}
\hop{\hopcolumbus}{15.5~\%}{60~min}{\oztog{1}}
\hop{\hopwarrior}{17~\%}{30~min}{\oztog{0.5}}
\hop{\hopcentennial}{10.5~\%}{10~min}{\oztog{1}}
\hop{\hopwarrior}{17~\%}{\dryh{}{7~days}}{\oztog{0.5}}
\hop{\hopcentennial}{10.5~\%}{\dryh{}{7~days}}{\oztog{0.5}}
\end{hops}

\begin{yeasts}
\yeast{Wyeast 1056}
\end{yeasts}

\end{ingredientsblock}

\chapter*{Biloxi Brewing Company Pale Ale}

\begin{aboutblock}
This is Biloxi Brewing Company's sessionable pale ale that bursts with grapefruit
flavors and aromas from the generous amount of Citra hops used, especially in the
dry hop addition.
\end{aboutblock}

\specifications{\styleamericanpaleale}{\galtol{5}}{1.049}{1.009}{5.2~\%}{37}{}{60~min}

\begin{methodandtiming}
 
\begin{mashsteps}
\mashstep{\ftoc{150}}{75~min}
\mashstep{\ftoc{168}}{10~min}
\end{mashsteps}

\begin{fermentationsteps}
\fermentationstep{\ftoc{67}}{}
\end{fermentationsteps}

\begin{directions}
Age at \ftoc{65} for 30 days.
\end{directions}

\end{methodandtiming}

\pagebreak

\begin{ingredientsblock}

\begin{malts}
\malt{Pale Two-row}{\lbtokg{6.47}}
\malt{Munich}{\lbtokg{1}}
\malt{Carapils}{\oztokg{12.4}}
\malt{Crystal 40 L}{\oztokg{12.4}}
\end{malts}

\begin{hops}
\hop{\hopcitra}{14~\%}{60~min}{\oztog{0.42}}
\hop{\hopcascade}{6.3~\%}{15~min}{\oztog{0.42}}
\hop{\hopcitra}{12.2~\%}{15~min}{\oztog{0.42}}
\hop{\hopcascade}{}{\whirl{}{}}{\oztog{0.42}}
\hop{\hopcitra}{}{\whirl{}{}}{\oztog{0.42}}
\hop{\hopcitra}{}{\dryh{}{}}{\oztog{0.83}}
\hop{\hopcascade}{}{\dryh{}{}}{\oztog{0.42}}

\end{hops}

\begin{yeasts}
\yeast{Fermentis SafAle US-05}
\end{yeasts}

\end{ingredientsblock}

\chapter*{Indeed Brewing Co. Day Tripper Pale Ale}

\begin{aboutblock}
Day Tripper is a West Coast-style pale ale out of Minneapolis. Indeed Brewing describes Day
Tripper as having a heady, dank, citrus-laced aroma supported by a complex and subtly sweet
malt backbone.
\end{aboutblock}

\specifications{\styleamericanpaleale}{\galtol{5}}{1.052}{}{5.4~\%}{}{}{90~min}

\begin{methodandtiming}
 
\begin{mashsteps}
\mashstep{\ftoc{150}}{60~min}
\end{mashsteps}

\begin{fermentationsteps}
\fermentationstep{\ftoc{67}}{}
\end{fermentationsteps}

\begin{directions}
Age one week in secondary. Keg, or bottle with \oztog{5} priming sugar and condition
for 2 weeks.
\end{directions}

\end{methodandtiming}

\pagebreak

\begin{ingredientsblock}

\begin{malts}
\malt{Briess Pale Ale}{\lbtokg{5.5}}
\malt{Maris Otter}{\lbtokg{3.5}}
\malt{White Wheat}{\lbtokg{0.75}}
\malt{Briess Caramel 20 L}{\lbtokg{0.5}}
\malt{Briess Carapils}{\lbtokg{0.5}}
\malt{Briess Bonlander Munich 10 L}{\lbtokg{0.25}}

\end{malts}

\begin{hops}
\hop{\hopwillamette}{}{\fwh}{\oztog{0.25}}
\hop{\hopcascade}{}{20~min}{\oztog{1.5}}
\hop{\hopcascade}{}{10~min}{\oztog{1.5}}
\hop{\hopcolumbus}{}{10~min}{\oztog{0.5}}
\hop{\hopsummit}{}{10~min}{\oztog{0.5}}
\hop{\hopcascade}{}{\foh}{\oztog{1.5}}
\hop{\hopcolumbus}{}{\foh}{\oztog{1.5}}
\hop{\hopsummit}{}{\foh}{\oztog{1}}
\hop{\hopcascade}{}{\dryh{}{3~days}}{\oztog{1.5}}
\hop{\hopcolumbus}{}{\dryh{}{3~days}}{\oztog{1.5}}
\hop{\hopsummit}{}{\dryh{}{3~days}}{\oztog{1}}

\end{hops}

\begin{yeasts}
\yeast{Fermentis SafAle US-05 / Wyeast 1272}
\end{yeasts}

\end{ingredientsblock}

\chapter*{Lynnwood Brewing Drop Bear APA}

\begin{aboutblock}
Drop Bear APA from Lynnwood Grill \& Brewing (Raleigh, N.C.) is brewed with Mosaic,
El Dorado, and Galaxy hops that are nicely balanced by a mix of two-row, Munich,
and caramel malts.
\end{aboutblock}

\specifications{\styleamericanpaleale}{\galtol{5.5}}{}{}{5.5~\%}{}{\srmtoebc{5.3}}{60~min}

\begin{methodandtiming}
 
\begin{mashsteps}
\mashstep{\ftoc{153}}{60~min}
\end{mashsteps}

\begin{fermentationsteps}
\fermentationstep{\ftoc{67}}{}
\end{fermentationsteps}

\begin{directions}
Sulfate to Chloride at 1:1.
\end{directions}

\end{methodandtiming}

\pagebreak

\begin{ingredientsblock}

\begin{malts}
\malt{Great Western Premium Tow-row}{\lbtokg{11}}
\malt{Briess Bonlander Munich 10 L}{\oztokg{6}}
\malt{Briess Caramel 20 L}{\oztokg{4}}
\malt{Briess Caramel 40 L}{\oztokg{4}}
\end{malts}

\begin{hops}
\hop{\hopmosaic}{12.25~\%}{15~min}{\oztog{0.5}}
\hop{\hopeldorado}{15~\%}{10~min}{\oztog{0.7}}
\hop{\hopgalaxy}{14~\%}{5~min}{\oztog{0.7}}
\hop{\hopeldorado}{15~\%}{\whirl{}{}}{\oztog{0.8}}
\hop{\hopgalaxy}{14~\%}{\whirl{}{}}{\oztog{0.8}}
\hop{\hopmosaic}{12.25~\%}{\whirl{}{}}{\oztog{0.8}}
\hop{\hopgalaxy}{14~\%}{\dryh{}{}}{\oztog{3}}
\hop{\hopeldorado}{15~\%}{\dryh{}{}}{\oztog{1.5}}
\hop{\hopmosaic}{12.25~\%}{\dryh{}{}}{\oztog{1.5}}
\end{hops}

\begin{yeasts}
\yeast{Fermentis SafAle S04 / Wyeast 1056 / White Labs WLP001}
\end{yeasts}

\end{ingredientsblock}

\chapter*{Maine Beer Co. Peeper Ale}

\begin{aboutblock}
Peeper Pale Ale was the first recipe the Kleban brothers perfected when they decided
to open their brewery in 2009. Maine Beer Co. (Freeport, Maine) describes Peeper as dry,
clean, and well balanced, with a generous dose of American hops.
\end{aboutblock}

\specifications{\styleamericanpaleale}{\galtol{5}}{1.047}{1.007}{}{}{}{60~min}

\begin{methodandtiming}
 
\begin{mashsteps}
\mashstep{\ftoc{150}}{}
\end{mashsteps}

\begin{fermentationsteps}
\fermentationstep{\ftoc{68}}{}
\end{fermentationsteps}

\end{methodandtiming}

\pagebreak

\begin{ingredientsblock}

\begin{malts}
\malt{Pale Two-row}{\lbtokg{8}}
\malt{Vienna}{\lbtokg{0.5}}
\malt{Red Wheat}{\oztokg{6}}
\malt{Carapils}{\oztokg{6}}
\end{malts}

\begin{hops}
\hop{\hopmagnum}{10~\%}{60~min}{\oztog{0.25}}
\hop{\hopamarillo}{9.2~\%}{10~min}{\oztog{0.15}}
\hop{\hopcascade}{5.5~\%}{10~min}{\oztog{0.2}}
\hop{\hopcentennial}{10~\%}{10~min}{\oztog{0.2}}
\hop{\hopamarillo}{9.2~\%}{\whirl{}{}}{\oztog{1.5}}
\hop{\hopcascade}{5.5~\%}{\whirl{}{}}{\oztog{2}}
\hop{\hopcentennial}{10~\%}{\whirl{}{}}{\oztog{2}}
\hop{\hopamarillo}{9.2~\%}{\dryh{}{}}{\oztog{2.5}}
\hop{\hopcentennial}{10~\%}{\dryh{}{}}{\oztog{2.5}}
\end{hops}

\begin{yeasts}
\yeast{Wyeast 1056}
\end{yeasts}

\end{ingredientsblock}

\chapter*{Maplewood Brewing Co. Charlatan American Pale Ale}

\begin{aboutblock}
Charlatan is hopped with Simcoe, Citra, and Centennial, and this pale ale is packed with
tropical flavors like mango, passionfruit, and grapefruit according to Chicago's Maplewood
Brewery.
\end{aboutblock}

\specifications{\styleamericanpaleale}{\galtol{5}}{1.059}{1.012}{6.1~\%}{32}{\srmtoebc{5.1}}{90~min}

\begin{methodandtiming}
 
\begin{mashsteps}
\mashstep{\ftoc{152}}{45~min}
\end{mashsteps}

\begin{directions}
Target mash pH 5.3. Sparge with \ftoc{170} water. At knockout, use a large spoon or stirrer
and create a whirlpool to cool the wort for a few minutes. The closer you can get to \ftoc{190}
the better, but anything under \ftoc{200} will work. Add the whirlpool hops and let rest 10
minutes, then whirlpool again for a few minutes to create a nice compact trub / hop pile.
When the fermentation is about 1~°P from FG, add the dry hops. After 3--7 days after FG, crash
and rack off of the hops. Carbonate to 2.55 volumes \ce{CO2}.

Water Profile: use Gypsum and Calcium Chloride to adjust to: 115--125 ppm Calcium,
65--75 ppm Chloride, 145--155 ppm Sulfate. Finished \ce{SO4}/\ce{Cl} ratio: 2.3--2.5.
Use Lactic Acid (88~\% strength) to adjust water pH, we use a 0.55~ml/gal rate, and we
adjust both the mash water and sparge water separately (based on the volume of each).
If you do not have the capability to adjust mash and sparge water separately, you may
make all of the mineral / acid additions based on your mash water volume. Always use
de-chlorinated water.
\end{directions}

\end{methodandtiming}

\pagebreak

\begin{ingredientsblock}

\begin{malts}
\malt{Rahr Standard Two-row}{\lbtokg{8}}
\malt{Weyermann Munich I}{\lbtokg{0.75}}
\malt{Crisp Dextrin}{\lbtokg{0.33}}
\malt{Crisp Cara}{\lbtokg{0.33}}
\malt{Rahr White Wheat}{\lbtokg{0.33}}
\end{malts}

\begin{hops}
\hop{\hopcolumbus}{16.5~\%}{90~min}{\oztog{0.1}}
\hop{\hopsimcoe}{13~\%}{15~min}{\oztog{0.2}}
\hop{\hopcitra}{12.5~\%}{10~min}{\oztog{0.1}}
\hop{\hopcentennial}{10~\%}{10~min}{\oztog{0.1}}
\hop{\hopsimcoe}{13~\%}{10~min}{\oztog{0.2}}
\hop{Whirlfloc Tablet}{}{10~min}{1}
\hop{Yeast Nutrient}{}{10~min}{1}
\hop{\hopcitra}{12.5~\%}{\whirl{}{10~min}}{\oztog{0.33}}
\hop{\hopcentennial}{10~\%}{\whirl{}{10~min}}{\oztog{0.33}}
\hop{\hopsimcoe}{13~\%}{\whirl{}{10~min}}{\oztog{0.165}}
\hop{\hopcitra}{12.5~\%}{\dryh{}{3~days}}{\oztog{1.25}}
\hop{\hopcentennial}{10~\%}{\dryh{}{3~days}}{\oztog{0.75}}
\hop{\hopsimcoe}{13~\%}{\dryh{}{3~days}}{\oztog{0.75}}
\end{hops}

\begin{yeasts}
\yeast{Wyeast 1318}
\end{yeasts}

\end{ingredientsblock}

\part{\styleamericanstout}

\chapter*{Rogue Chocolate Stout Clone}

\begin{aboutblock}
Many chocolate stouts seem to have acidic coffee flavors and end up falling short of real
chocolate flavor. However, Rogue’s Chocolate Stout is smooth, sweet and velvety, and tastes
like a chocolate malted milkshake with slight hoppiness and bittersweet chocolate notes.
The allure of this beer comes down to the chocolate flavors, which makes it enjoyable for
both the chocolate and beer lover. The sweet, creamy head gives way to a rich chocolate
truffle finish. This beer pairs perfectly with desserts as it's balanced by plenty of roasty
flavor and bitterness.
\end{aboutblock}

\specifications{\styleamericanstout}{\galtol{5}}{1.069}{1.017}{6.8~\%}{30}{\srmtoebc{25}}{90~min}

\begin{methodandtiming}
 
\begin{mashsteps}
\mashstep{\ftoc{150}}{60~min}
\end{mashsteps}

\begin{fermentationsteps}
\fermentationstep{\ftoc{60}}{7~days}
\fermentationstep{\ftoc{50}}{--}
\end{fermentationsteps}

\begin{directions}
Siphon into secondary at \ftoc{50} onto chocolate extract and hold until fermentation is
complete, then package and condition.
\end{directions}

\end{methodandtiming}

\pagebreak

\begin{ingredientsblock}

\begin{malts}
\malt{Great Western Premium Tow-row}{\lbtokg{11}}
\malt{Crystal 120 L}{\lbtokg{0.5}}
\malt{Chocolate}{\lbtokg{0.5}}
\malt{Rolled Oats}{\lbtokg{0.5}}
\malt{Roast Barley}{\oztokg{3}}
\end{malts}

\begin{hops}
\hop{\hopcascade}{5~\%}{90~min}{\oztog{1}}
\hop{\hopcascade}{5~\%}{30~min}{\oztog{1}}
\hop{Irish Moss}{}{20~min}{\tsptoml{1}}
\hop{\hopcascade}{5~\%}{\foh}{\oztog{1}}
\end{hops}

\begin{yeasts}
\yeast{Wyeast 1764-PC / White Labs WLP001}
\end{yeasts}

\begin{twists}
\twist{Chocolate Extract}{Secondary}{\oztog{1.51}}
\end{twists}

\end{ingredientsblock}

\chapter*{Stout Trousers}

\begin{aboutblock}
Chris O'Brien is a homebrewer with a mission: How to Drink
Beer and Save the World, O'Brien takes homebrewing to the next level
by promoting the use of responsibly cultivated, organic ingredients.
"Making organic homebrew is no different than making regular homebrew.
Just start with fresh, organic ingredients and the rest is the art and
science of homebrewing -- same as usual. The one big challenege is finding those ingredients." Of course, if you can't get your hands on strictly
organic ingredients, Stout Trousers -- a deliciously hoppy American
stout -- can still be brewed with non-organic substitutes.
\end{aboutblock}

\specifications{\styleamericanstout}{\galtol{5}}{1.072}{1.018}{7.1~\%}{}{}{100~min}

\begin{methodandtiming}
 
\begin{mashsteps}
\mashstep{\ftoc{154}}{60~min}
\end{mashsteps}

\begin{fermentationsteps}
\fermentationstep{\ftoc{64}}{1~day}
\fermentationstep{\ftoc{66}}{until 75~\% fermented}
\fermentationstep{\ftoc{70}}{1~week}
\fermentationstep{\ftoc{40}}{2~days}
\end{fermentationsteps}

\begin{directions}
Do not use soft water for this brew. Add calcium if you are using reverse
osmosis or spring water. Acidify the strike water to a pH of 4.8. This should
result in a mash pH of 5.2. Also acidify the sparge water to 4.8.
\end{directions}

\end{methodandtiming}

\begin{ingredientsblock}

\begin{malts}
\malt{Great Western Organic Premium Two-row}{\lbtokg{11.5}}
\malt{Briess Organic Roasted Barley}{\lbtokg{0.75}}
\malt{Briess Organic Chocolate}{\oztokg{8.8}}
\malt{Briess Organic Caramel 60 L}{\oztokg{8.8}}
\malt{Weyermann CARAFA II}{\oztokg{3.2}}
\end{malts}

\begin{hops}
\hop{\hoppacificgem}{}{60~min}{\oztog{0.75}}
\hop{\hopfuggles}{}{15~min}{\oztog{1.5}}
\hop{\hopcascade}{}{10~min}{\oztog{1.5}}
\hop{Irish Moss}{}{10~min}{\tsptoml{1}}
\hop{\hopcascade}{}{5~min}{\oztog{1}}
\end{hops}

\begin{yeasts}
\yeast{Wyeast 1272}
\end{yeasts}

\end{ingredientsblock}

\part{\styleamericanwheatorrye}

\chapter*{Atlas Brew Works Rowdy Rye Ale}

\begin{aboutblock}
Atlas Brew Works in Washington D.C. created this hop-forward rye ale that is both
fun and aggressive. Using a large amount of specialty malts and three hop varieties,
it is sure to get rye ale lovers' attention.
\end{aboutblock}

\specifications{\styleamericanwheatorrye}{\galtol{5}}{1.057}{}{6.2~\%}{50}{\srmtoebc{19}}{90~min}

\begin{methodandtiming}
 
\begin{mashsteps}
\mashstep{\ftoc{154}}{}
\end{mashsteps}

\begin{directions}
Optional: Use carbon filtered DC tap water, add 1.5~g of calcium sulfate and
1.5~g calcium chloride to the mash, as well as 2 ml of food grade 85~\%
phosphoric acid to the sparge.
\end{directions}

\end{methodandtiming}

\pagebreak

\begin{ingredientsblock}

\begin{malts}
\malt{Briess Pilsen}{\lbtokg{6}}
\malt{Briess Rye}{\lbtokg{1.5}}
\malt{Weyermann CARARED}{\lbtokg{0.8}}
\malt{Briess Aromatic Munich}{\lbtokg{0.8}}
\malt{Briess Victory}{\lbtokg{0.8}}
\malt{Briess Aromatic Munich}{\lbtokg{0.4}}
\malt{Briess Carapils}{\lbtokg{0.4}}
\malt{Briess Midnight Wheat}{\lbtokg{0.2}}
\end{malts}

\begin{hops}
\hop{\hopbravo}{}{90~min}{\oztog{0.32}}
\hop{\hopzythos}{}{20~min}{\oztog{0.48}}
\hop{\hopcentennial}{}{5~min}{\oztog{1.23}}
\hop{\hopcentennial}{}{\dryh{}{}}{\oztog{0.63}}
\hop{\hopzythos}{}{\dryh{}{}}{\oztog{0.63}}
\end{hops}

\begin{yeasts}
\yeast{American Ale}
\end{yeasts}

\end{ingredientsblock}

\chapter*{Triton Brewing Co. Fieldhouse Wheat}

\begin{aboutblock}
This American wheat ale from Triton Brewing Company is a 2017 Great American
Beer Festival bronze medal winner from Indiana. The golden color, white head
and crisp flavor will satisfy any wheat lover's cravings!
\end{aboutblock}

\specifications{\styleamericanwheatorrye}{\galtol{10}}{1.055}{1.028}{5~\%}{25}{\srmtoebc{4.1}}{90~min}

\begin{methodandtiming}
 
\begin{mashsteps}
\mashstep{\ftoc{154}}{}
\end{mashsteps}

\begin{fermentationsteps}
\fermentationstep{\ftoc{68}}{}
\end{fermentationsteps}

\end{methodandtiming}

\pagebreak

\begin{ingredientsblock}

\begin{malts}
\malt{Two-row}{\lbtokg{15.5}}
\malt{White Wheat}{\lbtokg{5}}
\malt{Flaked Barley}{\oztokg{8}}
\end{malts}

\begin{hops}
\hop{\hopgoldings}{4.8~\%}{50~min}{\oztog{0.5}}
\hop{\hopliberty}{4.5~\%}{50~min}{\oztog{0.6}}
\hop{\hopfalconersflight}{11.3~\%}{20~min}{\oztog{0.5}}
\hop{\hopfalconersflight}{11.3~\%}{5~min}{\oztog{0.5}}
\hop{\hopfalconersflight}{11.3~\%}{\dryh{}{6~days}}{\oztog{0.9}}
\end{hops}

\begin{yeasts}
\yeast{Fermentis SafAle US-05}
\end{yeasts}

\end{ingredientsblock}


\chapter*{Überbrew White Noise American Wheat Ale}

\begin{aboutblock}
White Noise from Uberbrew in Billings, Mont. took home a gold medal at the 2016
Great American Beer Festival, adding to a medal count that helped them take home
the award for the 2016 Small Brewing Company and Small Brewing Company Brewer of
the Year.
\end{aboutblock}

\specifications{\styleamericanwheatorrye}{\galtol{5}}{1.055}{}{}{13}{}{90~min}

\begin{methodandtiming}
 
\begin{mashsteps}
\mashstep{\ftoc{149}}{}
\end{mashsteps}

\end{methodandtiming}

\pagebreak

\begin{ingredientsblock}

\begin{malts}
\malt{Weyermann Bohemian Pilsner}{\lbtokg{6.1}}
\malt{Weyermann Pale Wheat}{\lbtokg{4}}
\end{malts}

\begin{hops}
\hop{\hopperle}{8.4~\%}{60~min}{8~g}
\hop{\hopliberty}{4.9~\%}{30~min}{5~g}
\hop{\hopliberty}{4.9~\%}{5~min}{16~g}
\end{hops}

\begin{yeasts}
\yeast{American Wheat}
\end{yeasts}

\end{ingredientsblock}

\part{\stylebalticporter}

\chapter*{Continental-Style Baltic Porter}

\begin{aboutblock}
In the mid-1800s, Baltic porter as a style relieved a formative transformation:
lager yeast. Cold shipping of the first strong porter exports had mellowed the
English-brewed porters, but beers brewed in the Baltic countries needed yeast
adapted to ferment cool, not just condition cold. Lager yeast therefore became
the Baltic brewery standard; ale yeasts were unsuitable, and strong porters
lost much of their ale yeast-derived ester and phenols, gaining signature
lager smoothness.
\end{aboutblock}

\specifications{\stylebalticporter}{\galtol{5.5}}{1.077}{1.012}{8.6~\%}{29}{\srmtoebc{29}}{120~min}

\begin{methodandtiming}
 
\begin{mashsteps}
\mashstep{\ftoc{148}}{60~min}
\mashstep{\ftoc{168}}{mashout}
\mashstep{\ftoc{170}}{sparge}
\end{mashsteps}

\begin{fermentationsteps}
\fermentationstep{\ftoc{48}}{36~hours / fermentation start}
\fermentationstep{\ftoc{50}}{2~weeks}
\fermentationstep{\ftoc{55}}{free-raise to}
\fermentationstep{\ftoc{55}}{until fermentation slowndown}
\fermentationstep{\ftoc{60}}{7--14~days until terminal gravity}
\end{fermentationsteps}

\begin{directions}
Sparge at \ftoc{170}. After the boil, stir wort vigorously to create a
whirlpool and precipitate out the trub. Chill wort to \ftoc{48} as quickly
as possible. Cold condition for at least 1 month at \ftoc{35} before
packaging, although 3 months is better.
\end{directions}

\end{methodandtiming}

\begin{ingredientsblock}

\begin{malts}
\malt{Pilsner}{\lbtokg{5}}
\malt{Munich}{\lbtokg{4.25}}
\malt{Vienna}{\lbtokg{4}}
\malt{Weyermann CARAMUNICH I}{\lbtokg{1}}
\malt{Dingemans Special B}{\oztokg{8}}
\malt{Briess Extra Special}{\oztokg{8}}
\malt{Weyermann CARAFA II}{\oztokg{0.5}}
\end{malts}

\begin{hops}
\hop{\hopmarynka}{10.5~\%}{60~min}{\oztog{0.5}}
\hop{\hoplubelska}{5~\%}{60~min}{\oztog{2}}
\hop{Whirlfloc Tablet}{}{15~min}{1}
\end{hops}

\begin{yeasts}
\yeast{White Labs WLP802}
\end{yeasts}

\end{ingredientsblock}

\part{\styleaelgianandfrenchale}

\chapter*{Ladyface Ale Companie La Grisette}

\begin{aboutblock}
Ladyface Ale Companie brews their award-winning Belgian, French and American
ales in the Agoura Hills of California. La Grisette is a thirst-quenching
Belgian farmhouse ale akin to saison, which took silver in the 2018 Las Angeles
International Beer Competition.
\end{aboutblock}

\specifications{\styleaelgianandfrenchale}{\galtol{5}}{1.051}{}{5~\%}{28}{\srmtoebc{4}}{60~min}

\begin{methodandtiming}
 
\begin{mashsteps}
\mashstep{\ftoc{150}}{}
\end{mashsteps}

\end{methodandtiming}

\pagebreak

\begin{ingredientsblock}

\begin{malts}
\malt{Pale Two-row}{\lbtokg{3.8}}
\malt{Pale Wheat}{\lbtokg{3.8}}
\malt{Acidulated}{\lbtokg{1.25}}
\malt{Flaked Oats}{\lbtokg{1.12}}
\end{malts}

\begin{hops}
\hop{\hopapollo}{18.5~\%}{60~min}{\oztog{0.1}}
\hop{\hopapollo}{18.5~\%}{20~min}{\oztog{0.5}}
\end{hops}

\begin{yeasts}
\yeast{Brewing Science Institute B-22}
\end{yeasts}

\end{ingredientsblock}

\chapter*{Ozark Beer Co. Belgian Golden Ale}

\begin{aboutblock}
Ozark Beer Co. brews this traditional Belgian-style golden ale with pilsen malt
and noble hops that lend aromas of apples and pears. The perfectly timed hop
additions makes it a light, dry and very refreshing homebrew.
\end{aboutblock}

\specifications{\styleaelgianandfrenchale}{\galtol{5}}{1.055}{1.004}{6.7~\%}{33}{\srmtoebc{3.4}}{75~min}

\begin{methodandtiming}
 
\begin{mashsteps}
\mashstep{\ftoc{152}}{}
\end{mashsteps}

\begin{fermentationsteps}
\fermentationstep{\ftoc{80}}{}
\end{fermentationsteps}

\begin{directions}
Carbonate to 2.6 volumes \ce{CO2}.
\end{directions}

\end{methodandtiming}

\pagebreak

\begin{ingredientsblock}

\begin{malts}
\malt{Gypsum}{5.7~g}
\malt{Briess Pilsner}{\lbtokg{6}}
\malt{Briess Pale Ale}{\lbtokg{2.33}}
\malt{Briess White Wheat}{\lbtokg{4.8}}
\malt{Briess Raw White Wheat}{\lbtokg{4.8}}
\malt{Weyerman Acidulated}{\lbtokg{4.8}}
\end{malts}

\begin{hops}
\hop{\hopsterling}{8.4~\%}{60~min}{\oztog{0.6}}
\hop{\hopsterling}{8.4~\%}{45~min}{\oztog{0.6}}
\hop{Calcium Chloride}{}{45~min}{1.3~g}
\hop{Yeast Nutrient}{}{15~min}{1.6~g}
\hop{Whirlfloc Tablet}{}{10~min}{1}
\hop{\hopsterling}{8.4~\%}{5~min}{\oztog{0.6}}
\end{hops}

\begin{yeasts}
\yeast{Omega Yeast OYL-500}
\end{yeasts}

\end{ingredientsblock}

\part{\stylebelgiandarkstrongale}

\chapter*{Easter Quad}

\begin{aboutblock}
Michael Tonsmeire has made a name for himself in the beer world -- especially
with sour beers. However, in his Brewer of the Week post, he shared a not-so-sour
Easter Quad recipe he brewed with his neighbor, a priest, for a church's Easter Vigil.
Tonsmeire was expecting a light wheat beer, but instead was excited when his neighbor
wanted to brew a Belgian Quad with a couple ingredients mentioned in the Bible
(pomegranate and cardamom). Needless to say, it turned out delicious.
\end{aboutblock}

\specifications{\stylebelgiandarkstrongale}{\galtol{10}}{1.082}{}{}{24}{\srmtoebc{23}}{70~min}

\begin{methodandtiming}
 
\begin{mashsteps}
\mashstep{\ftoc{152}}{60~min}
\end{mashsteps}

\begin{fermentationsteps}
\fermentationstep{\ftoc{63}}{3~days}
\fermentationstep{\ftoc{70}}{--}
\end{fermentationsteps}

\begin{directions}
Batch sparged with \ftoc{180}. Carbonate to 2.4 volumes \ce{CO2}.
\end{directions}

\end{methodandtiming}

\pagebreak

\begin{ingredientsblock}

\begin{malts}
\malt{Pale Two-row}{\lbtokg{33}}
\malt{Weyermann CARMUNICH I}{\lbtokg{2}}
\malt{Weyermann CARAFA Special II}{\lbtokg{0.38}}
\end{malts}

\begin{hops}
\hop{Table Sugar}{}{--}{\lbtokg{2}}
\hop{\hophallertauertradition}{6~\%}{60~min}{\oztog{2.5}}
\hop{Yeast Nutrient}{}{10~min}{\tsptog{0.5}}
\hop{Whirlfloc Tablet}{}{10~min}{1}
\hop{Cardamon Seeds}{}{\foh}{0.5~g}
\end{hops}

\begin{yeasts}
\yeast{White Labs WLP530}
\end{yeasts}

\begin{twists}
\twist{Pomegranate Molasses}{Secondary}{\lbtokg{2}}
\end{twists}

\end{ingredientsblock}

\part{Appendix}

\chapter*{Yeast}

\begin{yeastinfos}{Brewing Science Institute}
\yeastinfo{A-18}{London Ale III}
\yeastinfo{B-22}{LaChouffe}
\end{yeastinfos}

\begin{yeastinfos}{GigaYeast}
\yeastinfo{GY054}{Vermont IPA}
\end{yeastinfos}

\begin{yeastinfos}{Imperial Yeast}
\yeastinfo{A04}{Barbarian}
\end{yeastinfos}

\begin{yeastinfos}{Inland Island Yeast Lab.}
\yeastinfo{INIS-003}{Colorado IPA}
\end{yeastinfos}

\begin{yeastinfos}{Omega Yeast}
\yeastinfo{OLY-052}{DIPA Ale}
\yeastinfo{OYL-500}{Saisonstein's Monster}
\end{yeastinfos}

\begin{yeastinfos}{The Yeast Bay}
\yeastinfo{WLP4000}{Vermont Ale}
\yeastinfo{WLP4042}{Hazy Daze}
\end{yeastinfos}

\begin{yeastinfos}{White Labs}
\yeastinfo{WLP001}{California Ale}
\yeastinfo{WLP051}{California V Ale}
\yeastinfo{WLP095}{Burlington Ale}
\yeastinfo{WLP530}{Abbey Ale}
\yeastinfo{WLP802}{Czech Budejovice Lager}
\yeastinfo{WLP810}{San Francisco Lager}
\end{yeastinfos}

\begin{yeastinfos}{Wyeast}
\yeastinfo{1007}{German Ale}
\yeastinfo{1056}{American Ale}
\yeastinfo{1272}{American Ale II}
\yeastinfo{1318}{London Ale III}
\yeastinfo{1764-PC}{ROGUE Pacman}
\yeastinfo{2112}{California Lager}
\yeastinfo{2278}{Czech Pils}
\end{yeastinfos}

\end{document}