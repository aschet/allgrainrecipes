\stylesection{\stylemunichdunkel}

% -----------------------------------------------------------------------------
\begin{recipe}{Church Brew Works Pious Monk Dunkel}
% -----------------------------------------------------------------------------

\begin{aboutblock}
This dark-style lager from the Church Brew Works won a Great American Beer
Festival 2012 silver medal and is based on a 150-year-old recipe from Munich,
Germany. This version is slightly darker and surprisingly easy drinking with
a low ABV and crisp finish. \sourceaha
\end{aboutblock}

\specifications{\stylemunichdunkel}{\galtol{5}}{1.054}{1.012}{5.5~\%}{26}{\srmtoebc{16.5}}{90~min}{}

\begin{methodandtiming}
 
\begin{mashsteps}
\mashstep{\ftoc{140}}{Start}
\mashdecoctthick{}
\mashdecoctboil{}
\mashdecoctreturn{\ftoc{154}}{}

\end{mashsteps}

\begin{fermentationsteps}
\fermentationstep{\ftoc{50}}{Pitch}
\fermentationstep{\ftoc{52}}{Free raise to}
\fermentationstep{\ftoc{35}}{After diacetyl rest reduce to by 2.5~°C/day}
\end{fermentationsteps}

\begin{directions}
Lager at \ftoc{35} for 3 weeks.
\end{directions}

\end{methodandtiming}

\recipebreak

\begin{ingredientsblock}

\begin{malts}
\malt{Pilsner}{\lbtokg{4.16}}
\malt{Munich}{\lbtokg{4.16}}
\malt{Weyermann Melanoidin}{\oztog{12}}
\malt{Weyermann CARAFOAM}{\oztog{10}}
\malt{Weyermann CARAAROMA}{\oztog{6}}
\malt{Weyermann CARAFA II}{\oztog{3}}
\end{malts}

\begin{hops}
\hop{\hopperle}{8.1~\%}{90~min}{\oztog{0.7}}
\hop{\hoptettnang}{4.7~\%}{20~min}{\oztog{0.7}}
\end{hops}

\singleyeast{Augustiner Lager}

\begin{twists}
\twist{Yeast Nutrient}{Primary}{\tsptog{2}}
\end{twists}

\end{ingredientsblock}

\end{recipe}
