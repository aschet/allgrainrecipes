\stylesection{\stylebohemianpilsner}

% -----------------------------------------------------------------------------
\begin{recipie}{Meister Groll Bohemian Pils}
% -----------------------------------------------------------------------------

\begin{aboutblock}
Recipe courtesy of Jan Brücklmeier of Aurora, Ohio. This classic Bohemian Pilsner
uses a traditional triple-decoction mash. \sourceaha
\end{aboutblock}

\specifications{\stylebohemianpilsner}{\galtol{5.28}}{1.048}{1.012}{4.8~\%}{40}{\srmtoebc{3}}{110~min}{}

\begin{methodandtiming}
 
\begin{mashsteps}
\mashstep{\ftoc{100}}{30~min}
\mashdecoctthick{with 1/3 of mash}
\mashstep{\ftoc{151}}{20~min}
\mashstep{\ftoc{162}}{10~min}
\mashdecoctboil{10~min}
\mashdecoctreturn{\ftoc{151}}{10~min}
\mashdecoctthick{with 1/3 of mash}
\mashstep{\ftoc{162}}{10~min}
\mashdecoctboil{10~min}
\mashdecoctreturn{\ftoc{162}}{10~min}
\mashdecoctthin{with 1/3 of mash}
\mashdecoctboil{10~min}
\mashdecoctreturn{\ftoc{172}}{10~min}
\end{mashsteps}

\begin{fermentationsteps}
\fermentationstep{\ftoc{52}}{Until attenuated to \sgtop{1.016}}
\fermentationstep{\ftoc{52}}{Transfer to secondary; 1~week}
\fermentationstep{\ftoc{32}}{Reduce to}
\end{fermentationsteps}

\begin{directions}
Water adjustment: use soft water like that in Pilsen, \waterprofile{7}{2}{2}{5}{5}{}.
Use top pressure of 16~psi (1.1~bar) in secondary. Allow the beer to mature for at
least another 4 weeks before serving.
\end{directions}

\end{methodandtiming}

\begin{ingredientsblock}

\begin{malts}
\malt{Bohemian Pilsner}{\lbtokg{9.2}}
\malt{Acidulated}{\oztog{8}}
\end{malts}

\begin{hops}
\hop{\hopsaaz}{3.5~\%}{110~min}{\oztog{0.7}}
\hop{\hopsaaz}{3.5~\%}{45~min}{\oztog{1.8}}
\hop{\hopsaaz}{3.5~\%}{15~min}{\oztog{0.9}}
\end{hops}

\singleyeast{Wyeast 2206}

\end{ingredientsblock}

\end{recipie}
